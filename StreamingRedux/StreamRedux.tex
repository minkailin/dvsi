% !TEX TS-program = pdflatexmk

\documentclass[12pt, preprint,numberedappendix]{emulateapj}
%\documentclass[12pt, preprint]{aastex}

\newcommand\submitms{n}		% set to y to follow AAS ``ms'' names, etc.
\newcommand\bibinc{n}		% set to y if bib pasted in .tex, set to n to use bibtex


\usepackage{pdfsync}
\usepackage{subeqnarray}
\usepackage{natbib}


\bibliographystyle{apj}

\newcommand{\ie}{i.e.\ }
\newcommand{\eg}{e.g.\ }
\newcommand{\ep}{\epsilon}

%% MATHY SYMBOLS
\newcommand{\p}{\partial}
\newcommand{\nab}{\vc{\nabla}}
\newcommand{\xv}{\vc{x}}
\newcommand{\kv}{\vc{k}}
\newcommand{\brak}[1]{\langle #1\rangle} %put in brackets
\newcommand{\vcs}[1]{\mbox{\boldmath{$\scriptstyle{#1}$}}}
\newcommand{\vc}[1]{\mbox{\boldmath{$#1$}}}
\DeclareMathSymbol{\varOmega}{\mathord}{letters}{"0A}
\DeclareMathSymbol{\varSigma}{\mathord}{letters}{"06}
\DeclareMathSymbol{\varPsi}{\mathord}{letters}{"09}

%% LABEL HELPERS %%
\newcommand{\Eq}[1]{Equation\,(\ref{#1})}
\newcommand{\Eqs}[2]{Equations (\ref{#1}) and~(\ref{#2})}
\newcommand{\Eqss}[2]{Equations (\ref{#1})--(\ref{#2})}
\newcommand{\App}[1]{Appendix~\ref{#1}}
\newcommand{\Sec}[1]{Sect.~\ref{#1}}
\newcommand{\Chap}[1]{Chapter~\ref{#1}}
\newcommand{\Fig}[1]{Fig.~\ref{#1}}
\newcommand{\Figs}[2]{Figs.~\ref{#1} and \ref{#2}}
\newcommand{\Figss}[2]{Figs.~\ref{#1}--\ref{#2}} 
\newcommand{\Tab}[1]{Table \ref{#1}}

%% UNITS %%
\newcommand{\gcc}{\;\mathrm{g\; cm^{-3}}}
\newcommand{\gsc}{\;\mathrm{g\; cm^{-2}}}
\newcommand{\cm}{\; {\rm cm}}
\newcommand{\mm}{\; {\rm mm}}
%\newcommand{\ps}{\; {\rm s^{-1}}}
\newcommand{\km}{\; {\rm km}}
\newcommand{\au}{\; \varpi_{\rm AU}}
\newcommand{\AU}{\; {\rm AU}}
\def\K{\; {\rm K}}

%% PLANET FORMATION %%
\def \RH {R_\mathrm{H}}
\def \RB {R_\mathrm{B}}
\def \vH {v_\mathrm{H}}
\def \ve {v_\mathrm{esc}}
\def \Mi {M_\mathrm{iso}}

%% PLANETESIMAL / PARTICLE-GAS %%
\newcommand{\ts}{t_{\rm stop}}
\newcommand{\taus}{\tau_{\rm s}} 
\newcommand{\vsett}{v_{\rm sett}}
\newcommand{\vdr}{v_{\rm drift}}
\newcommand{\diff}{\nU_s}
\newcommand{\Sc}{\rm Sc}
\newcommand{\gs}{_{\rm g}}
\newcommand{\ps}{_{\rm p}}

%% COMPACT LIST%%
\newenvironment{packed_item}{
\begin{itemize}
  \setlength{\itemsep}{1pt}
  \setlength{\parskip}{0pt}
  \setlength{\parsep}{0pt}
}{\end{itemize}}

\begin{document}

\slugcomment{Draft Modified \today}


\shorttitle{Idea}
\shortauthors{Youdin }

\title{Recap of Linear Streaming Instability}
\author{Andrew N.\ Youdin}
\affil{Steward Observatory, University of Arizona}

%\begin{abstract}
%\end{abstract}

\section{Full, Compressible Two Fluid Treatment}
\subsection{Basic Equations}

Shearing box equations for particle continuity and momentum, vertically unstratified.
\begin{eqnarray}
  \frac{\p \rho_{\rm p}}{\p t} + \vc{W} \cdot \nab \rho_{\rm p}
     - \frac{3}{2} \varOmega x \frac{\p \rho_{\rm p}}{\p y}
     &=&  - \rho_{\rm p}\nab \cdot \vc{W} \, ,
  \label{eq:dustcontinuity}\\
  \frac{\p \vc{W}}{\p t} + (\vc{W}\cdot\nab)\vc{W} 
      - \frac{3}{2} \varOmega x \frac{\p \vc{W}}{\p y}
     & =& 2 \varOmega W_y \hat{\vc{x}} \nonumber \\
     &  & \hspace{-2.cm} - \frac{1}{2} \varOmega W_x \hat{\vc{y}}
      - \frac{1}{\ts}\left(\vc{W}-\vc{U}\right)\, .
  \label{eq:dusteqmot}
\end{eqnarray}
Shearing box equations for gas continuity and momentum.
\begin{eqnarray}
  \frac{\p \rho_{\rm g}}{\p t} + \vc{U}\cdot \nab \rho_{\rm g}
      - \frac{3}{2} \varOmega x \frac{\p \rho_{\rm g}}{\p y}
     & = & - \rho_{\rm g} \nab \cdot \vc{U} \, ,
  \label{eq:continuity}\\
  \frac{\partial \vc{U}}{\p t} + (\vc{U}\cdot\nab)\vc{U}
      - \frac{3}{2} \varOmega x \frac{\p \vc{U}}{\p y} 
      &=& 2 \varOmega U_y \hat{\vc{x}} \nonumber \\
      & & \hspace{-1.0cm}- \frac{1}{2} \varOmega U_x \hat{\vc{y}}
      - c_{\rm s}^2 \nab \ln \rho_{\rm g} \nonumber \\
      & & \hspace{-1.0cm} +  2 \eta  \varOmega^2 r \hat{\vc{x}}
      - \frac{\ep}{\ts}\left(\vc{U}-\vc{W}\right)
  \, . \label{eq:gaseqmot}
\end{eqnarray}
The equilibrium solutions are $U_z = W_z = 0$ and
\begin{eqnarray}
  {U_x \over \eta v_{\rm K}} &=& \,\,\, {2 \epsilon \taus \over D_{\ep \tau}} 
     \equiv \tilde{U}_x
      \, , \label{eq:NSHux} \\
  {U_y  \over \eta v_{\rm K}} &=&  -{1+ \epsilon + \taus^2 \over D_{\ep \tau}} 
     \equiv \tilde{U}_y
      \, , \label{eq:NSHuy} \\
  {W_x  \over \eta v_{\rm K}} &=& -\frac{2 \tau_{\rm s}}  {D_{\ep\tau}}  
     \equiv \tilde{W}_x
      \, , \label{eq:NSHWx} \\
  {W_y  \over \eta v_{\rm K}} &=& -\frac{ 1+\ep} {D_{\ep \tau}} 
      \equiv \tilde{W}_y
      \, . \label{eq:NSHWy}
\end{eqnarray}
where the drift denominator $D_{\ep \tau} = (1+\epsilon)^2+\tau_{\rm s}^2$. 
The equilibrium relative speeds are simply related to (shear modified) Coriolis forces as:
\begin{subeqnarray}
\Delta V_x &=& W_x - U_x = 2 W_y \taus  \\
%      &\equiv& S_x \tau_s \eta v_{\rm K}  \nonumber \\
\Delta V_y &=& W_y - U_y =  -{W_x \over 2} \taus  %\nonumber \\
%      &\equiv& S_y \taus^2 \eta v_{\rm K}
\end{subeqnarray}
  %We define order unity (or smaller if $\ep \gg1$) parameters $S_x$ and $S_y$. 

\subsection{Linear equations}
We consider linear perturbations to the equations of motion.  We adopt dimensionless variables for compactness with the units of $1 \over \varOmega$ for time and $\eta r$ for length and $\rho_{g,0}$ for density.  Explicitly we expand our dynamical variables in terms of equilibrium and perturbed terms as:
\begin{subeqnarray}
{W_x \over \eta v_{\rm K}} &=&  \tilde{W}_x + w_x \,\,\, ,\,\,\, {W_y \over \eta v_{\rm K}} = \tilde{W}_y + w_y\\
{U_x \over \eta v_{\rm K}} &=& \tilde{U}_x + u_x\,\,\, , \,\,\, {U_y \over \eta v_{\rm K}} = \tilde{U}_y + u_y\\
{W_z \over \eta v_{\rm K}} &=& w_z \,\,\, , \,\,\, {U_z \over \eta v_{\rm K}} = u_z\\
{\rho_p \over \rho_{\rm g0}} &=& \epsilon + \delta \,\,\, ,\,\,\, {\rho_p \over \rho_{\rm g0}} = 1 + \delta\\
{1 \over \Omega \ts} &=& {1 \over \taus} \left(1 + \delta_{\rm g} \right) 
\end{subeqnarray} 
The final perturbation to the stopping time assumes $\ts \propto 1 / \rho_{\rm g}$.  The details of this assumption are not important for (nearly) incompressible gas, the relevant case for protoplanetary disks.

Perturbed quantities are given a Fourier dependence $\propto \exp[ST + \imath(K_x X + K_z Z)]$, in terms of the normalized time $T$ and coordinates $X$ and $Z$.  The normalized complex growth rate $S = \Gamma - \imath \omega$ includes the pure growth term, $\Gamma$ and the oscillatory frequency $\omega$.	We also introduce $S_{\rm p} = S + \imath K_x W_x$ and $S_{\rm g} = S + \imath K_x U_x$ because these doppler shifted frequencies give more compact notation. The linearized equations of motion are
\begin{subeqnarray}
S_{\rm p} \delta &=&- \imath \epsilon(K_x w_x + K_z w_z)  \\
(S_{\rm p} + \taus^{-1}) w_x &=& 2 w_y + \tau_s^{-1} u_x - 2 W_y \delta_{\rm g} \\
(S_{\rm p} + \taus^{-1}) w_y &=& - w_x/2 + \taus^{-1} u_y + (W_x/2) \delta_{\rm g}\\
(S_{\rm p} + \taus^{-1}) w_z &=& \taus^{-1} u_z \\
S_{\rm g} \delta_{\rm g} &=& -\imath (K_x u_x + K_z u_z) \slabel{eq:gascont}\\
(S_{\rm g} + \ep \taus^{-1}) u_x &=& 2 u_y + \ep \taus^{-1} w_x - \imath K_x C^2 \delta_{\rm g} \nonumber \\
&&+ 2 W_y \delta \\
(S_{\rm g} + \ep \taus^{-1}) u_y &=& - u_x/2 + \ep \taus^{-1} w_y - (W_x/2) \delta \\
(S_{\rm g} + \ep \taus^{-1}) u_z &=& \ep \taus^{-1} w_z  - \imath K_z C^2 \delta_{\rm g}
\end{subeqnarray} 
where the tildes have been dropped from the mean velocities.  

\section{Incompressible Analysis}
The original and simpler case of the streaming instability is for incompressible gas, with $\nabla \cdot \vc{U} = 0$ from the outset.  In the linearized equations, we can identify $h \equiv C^2 \delta_{\rm g}$ as the (assumed order unity) enthalpy perturbation and then the incompressible limit $C\rightarrow \infty$ is equivalent to setting $\delta \rightarrow 0$ everywhere except the pressure terms.  Thus we replace \Eq{eq:gascont} with
\begin{equation}
K_x u_x + K_z u_z = 0
\end{equation} 
and modify the other equations containing $\delta_{\rm g}$ as described.  

An alternate formulation expressed the radial and horizonal motions in terms of their vorticity (curl) and divergence as
\begin{subeqnarray}
\Gamma &=& K_z w_x - K_x w_z \\
\Gamma_{\rm g} &=& K_z u_x - K_x u_z \\
\xi &=& K_x w_x + K_z w_z \\
\xi_{\rm g} &=& K_x u_x + K_z u_z = 0
\end{subeqnarray} 
With this formulation, note that $K^2 w_x = K_x \xi + K_z \Gamma$ with $K^2 = K_x^2 + K_z^2$.

The linearized equations are now reduced from 8 to 6 equations as
\begin{subeqnarray}
S_{\rm p} \delta &=&- \imath \epsilon \xi  \\
(S_{\rm p} + \taus^{-1}) \xi &=& 2 K_x w_y \\
(S_{\rm p} + \taus^{-1}) \Gamma &=& 2 K_z w_y + \taus^{-1} \Gamma_{\rm g}\\
(S_{\rm p} + \taus^{-1}) w_y &=& - {K_x \xi + K_z \Gamma \over 2 K^2} + \taus^{-1} u_y \\
(S_{\rm g} + \ep \taus^{-1}) \Gamma_{\rm g} &=& 2 K_z u_y + \ep \taus^{-1} \Gamma + 2 K_z W_y \delta \\
(S_{\rm g} + \ep \taus^{-1}) u_y &=& - {K_z \Gamma_{\rm g} \over 2K^2} + \ep \taus^{-1} w_y - {W_x \over 2} \delta 
\end{subeqnarray} 

\section{Center of Mass Formulation}
The streaming instability equations in YG05 were presented in terms of the center of mass and relative velocities
\begin{subeqnarray}
\vc{V} &=& {\rho_{\rm p} \vc{W} + \rho_{\rm g} \vc{U} \over \rho}\\
\Delta \vc{V} &=& \vc{W}
\end{subeqnarray} 
where $\rho = \rho_{\rm p} + \rho_{\rm g}$.  The perturbed variable for the COM problem are
\begin{subeqnarray}
\delta_\rho &=& {\delta \over 1+\ep}\\
\vc{v} &=& f_{\rm p} \vc{w} + f_{\rm g} \vc{u} + f_{\rm g}^2 \delta (\tilde{W} -  \tilde{U})\\
\Delta \vc{v} &=& \vc{w} - \vc{u}
\end{subeqnarray} 
where $f_{\rm g} = 1/(1+\ep)$ and $f_{\rm p} = \ep f_{\rm g}$.

%\begin{figure}[tb!] %  figure placement: here, top, bottom, or page
%\if\submitms y
% 	\includegraphics[width=6in]{f.eps}
% \else
%	\hspace{-1cm}
%   	\includegraphics[width=3.9in]{../figs/.pdf} 
%\fi
%   	\caption{}
%   	\label{fig:}
%\end{figure}

\acknowledgements
%Portions of this project were supported by the {\it NASA} {\it Astrophysics Theory Program} and  {\it Origins of Solar Systems Program}  through grant NNX10AF35G.\\

\if\bibinc n
\bibliography{refs}
\fi

\if\bibinc y
\begin{thebibliography}
\end{thebibliography}
\fi

\end{document}