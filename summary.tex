\section{Summary and discussion}
In this paper we highlight the similarity between an isothermal 
dusty gas and an adiabatic, pure gas. 
In the limit of perfect dust-gas coupling, the dust content of a gas 
parcel $\rhod/\rhog$ is conserved. This is analogous to entropy being 
conserved following a adiabatic, pure gas. %s = s(\rhod/\rhog)
We explicitly show that for 
a locally isothermal, dusty gas that the evolutionary equation of 
$\rhod/\rhog$ may be recast into an effective energy equation identical
in form to that in standard adiabatic hydrodynamics. We find the 
natural definition of the entropy of the dusty gas is 
\begin{align*}
  s  = \ln{\left(\frac{c_s^2\rhog}{\rhog+\rhod}\right)}.  
\end{align*}
This allows us to define the buoyancy frequency of an isothermal 
dusty gas. 

%
%finite coupling/heat source 

Finite dust-gas friction causes a relative drift between the two
components of the mixture. Then $\rhod/\rhog $ is no
longer conserved following a fluid parcel, as it is allowed to
exchange dust particles with neighboring parcels owing to a non-zero
stopping time $\tstop$. This is analogous to heat exchange between a 
pure gas parcel and its surroundings. Thus, finite dust-gas friction
may be interepreted as an energy/entropy source term for the dusty
gas.   

%
In this first study, we apply this equivalence principle to study the
axisymmetric stability of locally isothermal, protoplanetary disks
perfectly coupled to dust ($\tstop=0$). This is so that we may 
compute exact, steady equilibria for analyses. Furthermore,
in this limit the dusty problem is \emph{identical} to adiabatic, pure
gas dynamics. 

We obtain the Solberg-Hoiland criteria to assess
axisymmetric stability when the equation of state is strictly
isothermal. We show that {\bf dust-settling cannot lead to
  axisymmetric instability, however large the associated vertical
  shear} \citep[cf. \emph{non-axisymmetric} Kelvin-Helmholtz instabilities
  induced by  dust-settling, ][]{lee10}. We  also show that
protoplanetary disks cannot have sharp edges in the dust-to-gas ratio,
as this implies a sharp entropy gradient and may thus render the disk
unstable.  

%does not preclude non-axisymmetric instabilities. but if there is a
%global radial temp gradient, then vertical shear arising from
%it will always be comparable to dust-settling. but there is vsi with
%global rad temp grad 

For locally isothermal disks $c_s=c_s(r)$ we generalize the vertical
shear instability (VSI) to include dust. We find that dust-loading
generally has a stabilizing effect. In our disk models
dust-loading does not affect VSI growth rates significantly, but
meridional motions may be suppressed at heights where vertical 
buoyancy dominates over vertical shear. This confirms our 
identification of the entropy and buoyancy of an isothermal, dusty
gas.  

%growth rates not sig changed 
%v shear from thin dust layers are overwhelmed by 

\subsection{Dusty analogs of other gaseous instabilities} 
%gi
%rwi 

%\subsection{Implication for numerical simulations}


\subsection{Thermodynamic interpretation of other dusty instabilities}


%\subsection{Caveats and future directions}
%gi 
