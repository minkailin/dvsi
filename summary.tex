\section{Summary and discussion}\label{summary}
In this paper we highlight the similarity between an isothermal 
dusty gas and an adiabatic, pure gas. 
In the limit of perfect dust-gas coupling, the dust content of a fluid 
parcel $\rhod/\rhog$ is conserved. This is analogous to entropy being 
conserved following a adiabatic, pure gas. %s = s(\rhod/\rhog)
We explicitly show that for 
a locally isothermal, dusty gas that the evolutionary equation of 
$\rhod/\rhog$ may be recast into an effective energy equation identical
in form to that in standard adiabatic hydrodynamics. We find the 
natural definition of the entropy of the dusty gas is 
\begin{align*}
  s  = \ln{\left(\frac{c_s^2\rhog}{\rhog+\rhod}\right)}.  
\end{align*}
This allows us to define the buoyancy frequency of an isothermal 
dusty gas. 

%
%finite coupling/heat source 

Finite dust-gas friction causes a relative drift between the two
components of the mixture. Then $\rhod/\rhog $ is no
longer conserved following a fluid parcel, as it is allowed to
exchange dust particles with neighboring parcels owing to a non-zero
stopping time $\tstop$. This is analogous to heat exchange between a 
pure gas parcel and its surroundings. Thus, finite dust-gas friction
may be interepreted as an energy/entropy source term for the dusty
gas.   

%
In this first study, we apply this equivalence principle to study the
axisymmetric stability of locally isothermal, protoplanetary disks
perfectly coupled to dust ($\tstop=0$). This is so that we may 
compute exact, steady equilibria for analyses. Furthermore,
in this limit the dusty problem is \emph{identical} to adiabatic, pure
gas dynamics. 

We obtain the Solberg-Hoiland criteria to assess
axisymmetric stability when the equation of state is strictly
isothermal. We show that {\bf dust-settling cannot lead to
  axisymmetric instability, however large the associated vertical
  shear} \citep[cf. \emph{non-axisymmetric} Kelvin-Helmholtz instabilities
  induced by  dust-settling, ][]{lee10}. We also show that
sharp edges in the dust-to-gas ratio cannot persist in protoplanetary
disks, as these imply a sharp entropy gradient and may thus destablize
the disk. 

%does not preclude non-axisymmetric instabilities. but if there is a
%global radial temp gradient, then vertical shear arising from
%it will always be comparable to dust-settling. but there is vsi with
%global rad temp grad 

For locally isothermal disks $c^2_s=c_s^2(r)$ we generalize the vertical
shear instability (VSI) to include dust. We find that dust-loading
generally has a stabilizing effect. In our disk models
dust-loading does not affect VSI growth rates significantly, but
meridional motions may be suppressed at heights where vertical 
buoyancy dominates over vertical shear. This confirms our 
identification of the entropy and buoyancy of an isothermal, dusty
gas.  

%growth rates not sig changed 
%v shear from thin dust layers are overwhelmed by 

\subsection{Generalization to locally polytropic disks}
We can extend the dusty/adiabatic gas correspondence to 
other fixed equations of state. As an example, consider the locally
polytropic disk 
\begin{align}
  P = K(r,z)\rhog^{\Gamma}, 
\end{align}
where $K$ is a prescribed function and $\Gamma$ is the constant
adiabatic index. Then eliminating $\tepsilon$ from the dust equation
\ref{dusteq} gives 
\begin{align}
  \frac{D P}{D t} = - \Gamma P\nabla\cdot\bm{v}  + P \bm{v}\cdot\nabla \ln{K}
  + \frac{\Gamma P}{\rhog}\nabla\cdot\left(\tepsilon\tstop\nabla
  P\right).
\end{align}
Thus the dusty gas behaves like a pure gas with adiabatic index
$\Gamma$. The entropy is given by 
\begin{align}
  s = \ln{\left[K(r,z)\left(1 - \tepsilon\right)^\Gamma\right]}.  
\end{align}
When $K$ is constant, the corresponding Solberg-Hoiland criteria for
axisymmetric stability may be obtained from that in standard adiabatic 
hydrodynamics \citep[e.g.][]{tassoul78} by
using this definition of entropy in
Eqs. \ref{dusty_solberg1}---\ref{dusty_solberg2}. 

\subsection{Implications for numerical simulations}
The dusty/adiabatic gas equivalence means that pure
gas dynamics codes can be applied to simulate locally
isothermal/polytropic dusty gas with minimal modification. In fact, no
modification is needed  
for strictly isothermal or polytropic gas ($c_s$ or $K$ being
constant, respectively) perfectly-coupled to dust. The usual 
adiabatic energy equation is effectively a dust continuity
equation. When $c_s$ or $K$ is not constant and/or $\tstop\neq0$ one
can simply add corresponding source terms in energy equation. In the
large but finite coupling case, this source term is analogous to
radiative diffusion \citep{price15}, which is also standard in many
hydrodynamics codes.      

\subsection{Dusty analogs of other gaseous instabilities}  

%gi
%rwi 

\subsubsection{Gravitational instability} %vertical structure 
The addition of dust enhances gravitational
instability (GI). This is because dust particles contribute to the
total disk mass but not thermal pressure, which effectively lowers the
disk temperature \citep[][]{thompson88,shi13}. For typical dust-loading 
$\epsilon, Z\ll1$, this effect is unimportant. However, if $\epsilon$ is
large (e.g. due to dust settling) then the effective temperature may 
be lowered to enable instability. 

As noted in \S\ref{loc_iso_eos}, dust settling causes the effective
temperature to \emph{increase} away from the midplane. This contrasts
to previous studies of GI in vertically stratified disks
\citep[e.g.][]{mamat10, kim12,lin14c} where the temperature decreases
from the midplane. While we expect only the total surface density and
characteristic temperatures are relevant to stability  
\citep{toomre64}, having a non-trivial vertical temperature
structure, induced by dust, may modify the vertical structure of 3D
waves and unstable modes. We will examine this linear problem in a following 
study.  



%see also
%\S\ref{loc_iso_eos}
%We thus expect the standard Toomre
%parameter to be modified to $Q = c_s\sqrt{1-\tepsilon}\Omega/\pi
%G\Sigma$. 

\subsubsection{Rossby wave instability}
The Rossby wave instability (RWI) is a non-axisymmetric, 2D shear
instability that operates in thin disks when it has radial structure
\citep{lovelace99,li00}. These studies consider adiabatic pure gas and 
show instability is possible if there is an extrema in the quantity
\begin{align}
 \eta = \frac{\kappa^2}{2\Omega\Sigma}\mathcal{S}^{2/\gamma},  
\end{align} 
where $\mathcal{S} = \Pi/\Sigma^\gamma$. Here, $\Sigma$ is the total
surface density and $\Pi$ is the vertically integrated pressure. 

From the dusty/adiabatic gas equivalence, we deduce the corresponding
condition for RWI in a (locally isothermal or polytroipc) dusty gas is
also given by extrema in $\eta$ with $\mathcal{S} = \exp{s}$ (and
$\tepsilon\to \Sigma_\mathrm{d}/\Sigma$). Since
the dusty entropy $s$ reflects the dust distribution, we expect that
RWI may also be triggered by extrema in  the dust-to-gas ratio. 


%gap edges. thin dust rings. 

\subsection{Thermodynamic view of dusty instabilities}

%streaming instability 
Finite dust-gas drag can lead to the streaming instability (SI) in the
presence of a background radial pressure gradient \citep{youdin05a}.    
SI can also be recovered from the one-fluid framework
\citep{laibe14}. \cite{jacquet11} explains the essence of SI as dust
accumulation at pressure maxima (from terminal velocity
approximation), which back-reacts on the gas flow in 
such a way to strengthen said maxima. 

% Now consider our energy formulation of the
% dusty problem, Eq. \ref{eff_energy}. Assume constant $c_s$
% and that the dominant contribution to the energy source
% $\mathcal{C}$ is associated with $\nabla^2P$ \citep[which is
% responsible for SI, ][]{jacquet11}. Then 
% \begin{align*}
% \frac{DP}{Dt} \simeq \frac{P}{\rho}\frac{D\rho}{D t} + c_s^2\tepsilon \tstop
%   \nabla^2 P.
% \end{align*} 
% At pressure maxima, dust-drag acts as a cooling term. However, if
% dust accumulation into this maximum leads to a sufficiently large
% compression, there may be a net heating, enhancing the pressure
% maximum, and a positive feedback established.   

% This picture relies on unstable correlations between pressure
% perturbations and fluid compression, as pointed out by
% \cite{jacquet11}. Indeed, we see from Eq. \ref{thermal_instabilities}
% that instabilities due to drag are attributed to  
% \begin{align*}
%   \left(\nabla\cdot\dd\bm{v}^*\right) \delta \mathcal{C}.
% \end{align*} 
% Thus instabilities may arise if $\nabla\cdot\delta \bm{v}$ is suitably
% correlated to the perturbed drag force $\delta \mathcal{C}$ (which for
% SI is due to pressure perturbations). This is analogous  
% to thermal instabilities in gas dynamics where $\mathcal{C}$
% represents non-adiabatic heating/cooling and may lead to 

% \citep{latter06},  






%loren's instability 