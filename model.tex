\section{Single fluid description of well-coupled dusty gas}\label{setup} 

\cite{laibe14}, Eq. 84 --- 85, 87: 

\begin{align}
  &\frac{D\rho}{Dt} = -\rho\nabla\cdot\bm{v}, \label{masseq}\\ 
  &\frac{D\bm{v}}{Dt} = - \nabla\Phi - \frac{1}{\rho}\nabla
  P, \label{momeq}\\ 
  &\frac{D\epsilon}{Dt} = -\frac{1}{\rho} \nabla \cdot \left(\epsilon 
  \tstop \nabla P \right),\label{dusteq}  
\end{align}
where $D/Dt \equiv \p_t + \bm{v}\cdot\nabla$ is the Lagragian
derivative; 
\begin{align}
  \rho \equiv \rhog + \rhod
\end{align}
is the total density with $\rhog$ and $\rhod$ being the gas and dust
density, respectively; 
\begin{align}
  \bm{v} \equiv \frac{\rhog\bm{v}_\mathrm{g} +
    \rhod\bm{v}_\mathrm{d}}{\rho}
\end{align}
is the center-of-mass velocity with $\bm{v}_\mathrm{g}$ and
$\bm{v}_\mathrm{d}$ being the gas and dust velocities, respectively; 
and 
\begin{align}
  \epsilon = \frac{\rhod}{\rho} 
\end{align}
is the dust fraction; and $\tstop$ is the stopping time. 

\subsection{Locally isothermal equation of state for the gas} 

\begin{align}\label{eos}
  P = c_s^2(r,z)\rhog = c_s^{2}(1 - \epsilon)\rho,   
\end{align}

$c_s$ is a prescribed sound-speed profile fixed in time. 

\subsection{Effective energy equation for the mixture with a
  stationary temperature profile}   
Eliminate $\epsilon$ in favor of $P$,

\begin{align}
  \epsilon = 1 - \frac{P}{c_s^2(r,z)\rho}, 
\end{align}

Eq. \ref{dusteq} becomes

\begin{align}
\frac{DP}{Dt} = - P \nabla\cdot\bm{v} + P\bm{v}\cdot\nabla\ln{c_s^2} +
c_s^2 \nabla\cdot\left[\tstop\left(1 - \frac{P}{c_s^2\rho}\right)\nabla
  P\right]. \label{eff_energy}
\end{align}

This is the energy equation for an ideal gas of adiabatic index
$\gamma=1$, with additional source terms due to the imposed
temperature profile, as well as dust-gas friction.  

Eq. \ref{eff_energy} can be written in conservative form,

\begin{align*}
  \frac{\p P}{ \p t} + \nabla\cdot\left\{P\left[\bm{v} -
      \tstop\left(c_s^2 - \frac{P}{\rho}\right)\nabla\ln{P}\right]
    \right\}\\
  = \left[P\bm{v} - \tstop\left(c_s^2 - \frac{P}{\rho}\right)\nabla
    P\right]\cdot\nabla\ln{c_s^2}. 
\end{align*}
note: non-linear in $P$

\section{Equilibrium}

\begin{align}
  r\Omega^2 &= \frac{\p \Phi}{\p r} + \frac{1}{\rho}\frac{\p P}{\p
    r},\\
  0 & = \frac{\p\Phi}{\p z} + \frac{1}{\rho}\frac{\p P}{\p z}.
\end{align}

\subsection{Vertical shear}

\begin{align}\label{vshear}
  r\frac{\p \Omega^2}{\p z} &= \frac{\p\ln{\rho}}{\p r}\frac{\p}{\p
    z}\left[c_s^2(1-\epsilon)\right] - \frac{\p\ln{\rho}}{\p z}
  \frac{\p}{\p r} \left[c_s^2(1-\epsilon)\right]\\  
  & = \frac{1}{\rho}\left(\frac{\p P}{\p r}\frac{\p s}{\p z} -\frac{\p
    P}{\p z}\frac{\p s}{\p r} \right),\notag
\end{align}
where
\begin{align}
   s \equiv \ln \frac{P}{\rho}
\end{align}
may be interpreted as the mixture entropy.


\section{Limiting behaviours}

\subsection{Strictly isothermal gas perfectly coupled to dust}  
When $c_s^2$ is a constant and $\tstop=0$, the dusty gas equations are
exactly equivalent to that for adiabatic hydrodyamics with unit adiabatic
index. Although the gas is strictly isothermal, the mixture behaves 
adiabatically because the dust fraction $\epsilon$ is advected with
the gas (i.e. evolves like an entropy when there is no heating or
cooling). 

In this case one may immediately apply the Solberg-Hoiland
criteria to assess axisymmtric \emph{stability}:  

\begin{align}
  \kappa^2 - \frac{1}{\rho}\nabla P \cdot \nabla s &> 0,\\
  -\frac{1}{\rho}\frac{\p P}{\p z} \left(\kappa^2 \frac{\p s}{\p z} -
  r\frac{\p\Omega^2}{\p z}\frac{\p s}{\p r}\right) &>0. 
\end{align}  
Here we may take $s=\ln{1-\epsilon}$ from the definition of $s$ and
the equation of state (Eq. \ref{eos}) with constant sound-speed. 



%\begin{align}
%  s \equiv \ln \frac{P}{\rho} = \ln{(1-\epsilon)} 
%\end{align}
% The first equality is
%similar to the usual defintion for gas entropy (with unit adiabatic
%index). The second follows from the imposed equation of state,
%Eq. \ref{eos}, with a constant sound-speed. 

The stability conditions are thus
\begin{align}
  \kappa^2 + c_s^2 \nabla\ln{\rhog}\cdot\nabla\epsilon &> 0,  \\
  -c_s^2\frac{\p\ln{\rhog}}{\p z}\left(-\kappa^2\frac{\p\epsilon}{\p
    z} + r\frac{\p\Omega^2}{\p z}\frac{\p \epsilon}{\p r} \right) & > 0. 
\end{align}
For accretion disks, $\rhog$ and $\epsilon$ are expected to decrease
outwards and away from the midplane, implying 
$\nabla\rhog\cdot\nabla\epsilon>0$, so the first criterion is
typically satisifed. 

To evaulate the second criterion, we can consider $z>0$ without loss
of generality. Then $\p_z\rhog,\p_z\rhod <0$, so $\p_z\rho<0$. We
assume $\p_r\rho <0$ and $\p_z\epsilon, \p_r\epsilon <0$ as
well\footnote{Due to dust sedimentation and inwards migration.}. Then,
inserting the vertical shear profile, Eq. \ref{vshear}, we find the
second criterion becomes   

\begin{align} 
  \kappa^2\left|\frac{\p\epsilon}{\p z}\right| + c_s^2\left|\frac{\p
    \epsilon}{\p r}\frac{\p\epsilon}{\p z}\frac{\p\ln{\rho}}{\p
    r}\right| - c_s^2 \left(\frac{\p\epsilon}{\p
    r}\right)^2\left|\frac{\p\ln{\rho}}{\p z}\right| > 0.  \label{criterion2}
\end{align}

This equation shows that the vertical gradient in dust-fraction (and
thus the dust-to-gas ratio $\rhod/\rhog$) has a stabilizing effect for
axisymmetric perturbations. Instead, the only way to destabilize the
system is a sufficiently large radial gradient in the dust-fraction
and/or vertical (total) density gradient (the last term).  

Note that Eq. \ref{criterion2} is only applicable for the assume disk
structure described above: decreasing dust and gas densities, as well
as the dust-fraction, away from the star and away from the
midplane. For less typical disk structures, one must evalulate above the
critera seperately. 

\subsection{Locally isothermal gas perfectly coupled to dust} 
$c_s(r,z)$ is non-uniform, but $\tstop=0$. Integral relation for
linear, axisymmetric waves with frequency $\sigma$ (see Appendix \ref{var_prin})
\begin{align}
&  \sigma^2\int\rho\left(|\dd v_r|^2 + |\dd v_z|^2\right)dV \notag\\
&= \int\left[ \rho
  |\dd v_r|^2A + \rho  \dd v_z \dd v_r^* B + \rho \dd v_z^*\dd v_r C +
  \rho |\dd v_z|^2 D\phantom{\frac{1}{1}}\right. \notag\\
&\phantom{===}  \left. + \frac{1}{P}\left|\nabla\cdot\left(P\dd
  \bm{v}\right)\right|^2\right]dV - \int P
  \left(\nabla\cdot\dd\bm{v}^*\right)\left(\dd\bm{v}\cdot\nabla\ln{c_s^2}\right)dV.
\end{align}
The real coefficients $A,B,C,D$ can be read off
Eq. \ref{integral_ex}. If $c_s$ is constant, $\sigma$ is real, and 
the first integral leads to the usual Solberg-Hoiland criteria for
axisymmetric stability. However, when $c_s$ is non-uniform, $\sigma$
is generally complex, implying instability. This is the usual vertical
shear instability. 
