\section{Single fluid description of dusty gas}\label{setup} 
In the limit of strong dust-gas drag (stopping time $\tstop\to 0$) 
\cite{laibe14} give a single fluid description of the mixture:  
(their Eq. 84 --- 85 and Eq. 87): 

\begin{align}
  &\frac{D\rho}{Dt} = -\rho\nabla\cdot\bm{v}, \label{masseq}\\ 
  &\frac{D\bm{v}}{Dt} = - \nabla\Phi - \frac{1}{\rho}\nabla
  P, \label{momeq}\\ 
  &\frac{D\tepsilon}{Dt} = -\frac{1}{\rho} \nabla \cdot \left(\tepsilon 
  \tstop \nabla P \right),\label{dusteq}  
\end{align}
where $D/Dt \equiv \p_t + \bm{v}\cdot\nabla$ is the Lagragian
derivative; 
\begin{align}
  \rho \equiv \rhog + \rhod
\end{align}
is the total density with $\rhog$ and $\rhod$ being the gas and dust
density, respectively; 
\begin{align}
  \bm{v} \equiv \frac{\rhog\bm{v}_\mathrm{g} + 
    \rhod\bm{v}_\mathrm{d}}{\rho}
\end{align}
is the center-of-mass velocity with $\bm{v}_\mathrm{g}$ and
$\bm{v}_\mathrm{d}$ being the gas and dust velocities, respectively; 
and 
\begin{align}
  \tepsilon \equiv \frac{\rhod}{\rho}  = \frac{\epsilon}{1+\epsilon} 
\end{align}
is the dust-fraction and $\epsilon=\rhod/\rhog$ is the usual
dust-to-gas ratio. The gas pressure $P$ is given by the chosen 
equation of state described below. The external potential $\Phi$ is
that for a star of mass $M_*$ at the origin 
\begin{align}\label{thin_disk_potential}
  \Phi(r,z) =-\frac{GM_*}{\sqrt{r^2 + z^2}}\simeq
  -\frac{GM_*}{r}\left(1 - \frac{z^2}{2r^2}\right), 
\end{align}
where $G$ is the gravitational constant. The second equality is the 
thin-disk approximation for the disk potential, appropriate for
$|z|\ll r$. We shall adopt this approximate potential in order to
obtain analytic expressions for the disk equilibria.  

\subsection{Locally isothermal equation of state} 
We adopt 
\begin{align}\label{eos}
  P = c_s^2(r,z)\rhog = c_s^{2}(1 - \tepsilon)\rho,   
\end{align}
where $c_s$ is a prescribed sound-speed profile fixed in time. 
In numerical calculations we will consider vertically 
isothermal disks with \begin{align}\label{power_temp}
  c_s^2(r) \propto r^{q},
\end{align}
where $q$ is the power-law index for the disk temperature. For $q=0$
the disk is strictly isothermal. 

If we define the effective sound-speed $c_{s,\mathrm{eff}} = c_s\sqrt{\left(1 -
    \tepsilon\right)}$, then the equation of state has the standard
form $P=c_{s,\mathrm{eff}}^2\rho$. Since the dust-fraction $\tepsilon$
typically decrease away from the midplane, we expect vertically
isothermal dusty disks to in fact behave as if it
had a temperture \emph{increasing} away from $z=0$.   

\subsection{Effective energy equation}
We show that for a prescribed temperature distribution, the mixture
obeys an evolutionary energy equation, due to the advection of the
dust-fraction. Eliminate $\tepsilon$ in favor of $P$, 
\begin{align*}
  \tepsilon = 1 - \frac{P}{c_s^2(r,z)\rho}, 
\end{align*}
then Eq. \ref{dusteq} becomes
\begin{align}
\frac{DP}{Dt} &= - P \nabla\cdot\bm{v} + P\bm{v}\cdot\nabla\ln{c_s^2}
                + \mathcal{C},  \label{eff_energy} \\
\mathcal{C}&\equiv c_s^2 \nabla\cdot\left[\tstop\left(1 -
  \frac{P}{c_s^2\rho}\right)\nabla 
  P\right].
\end{align}
Eq. \ref{eff_energy} is the energy equation for an ideal gas of adiabatic index
$\gamma=1$, i.e. isothermal evolution, but now with additional energy source 
terms due to the imposed temperature profile via $\nabla c_s^2$, as
well as dust-gas drag via $\mathcal{C}$. 
%In this form, the function $\mathcal{C}$ can be interpreted
%as a cooling term.    

Eq. \ref{eff_energy} can be written in conservative form,
\begin{align*}
  \frac{\p P}{ \p t} + \nabla\cdot\left\{P\left[\bm{v} -
      \tstop\left(c_s^2 - \frac{P}{\rho}\right)\nabla\ln{P}\right]
    \right\}\\
  = \left[P\bm{v} - \tstop\left(c_s^2 - \frac{P}{\rho}\right)\nabla
    P\right]\cdot\nabla\ln{c_s^2}. 
\end{align*}
We may thus re-interpret $P$ as the mixture's energy density, but the
energy flux has an additional contribution from the pressure
gradient and dust-gas friction. The term on the right-hand-side, owing
to the imposed temperature profile, can be interpreted as an external
heat source.  

%We comment that 

\subsection{Entropy of an isothermal dusty gas }
%Having made the connection between the dust fraction (or dust-to-gas
%ratio) of the mixture and an energy equation, we now define 
We define the (dimensionless) entropy of the mixture as 
\begin{align}
   s \equiv \ln \frac{P}{\rho} = \ln{\left[c_s^2(1-\tepsilon)\right]}.  
\end{align}
We shall see that by defining the entropy this way, many of the 
results concerning the stability of dusty gas will have identical form
and interpretations to that in standard adiabatic hydrodynamics. 

\section{Dynamical equilibria}\label{eqm}
 
For a given distribution of the dust-fraction $\tepsilon$ (or
dust-to-gas ratio $\epsilon$), the 
mass and momentum Eqs. \ref{masseq}---\ref{momeq} admit     
axisymmetric steady states with $\rho(r,z)$ and 
$\bm{v}=r\Omega(r,z)\hat{\bm{\phi}}$ where $\Omega = v_\phi/r$, which satisfy 
\begin{align}
  r\Omega^2 &= \frac{\p \Phi}{\p r} + \frac{1}{\rho}\frac{\p P}{\p
    r},\label{steady_momr}\\
  0 & = \frac{\p\Phi}{\p z} + \frac{1}{\rho}\frac{\p P}{\p z},\label{steady_momz}
%  0 & = \nabla\cdot\left(\tepsilon\tstop\nabla P\right) \label{steady_dust}
\end{align}
with $P=P(\tepsilon,\rho)$ given by the equation of state
(Eq. \ref{eos}). An explicit solution is presented in
\S\ref{steady_state}.  
However, disk structures obtained this way generally do not satisfy 
the steady state effective energy Eq. \ref{eff_energy} 
except in the limit of perfectly coupled dust ($\tstop\to0$).  

True steady states may be calculated by invoking an underlying gas
turbulence, which leads to dust diffusion \citep{takeuchi02, youdin07, 
  lyra13}. This complication is beyond the scope of this paper and we 
defer it to a future work. Here, we will either consider $\tstop = 0$,
or assume that $\tstop$ is sufficiently small such that the background
disk (as obtained from Eq. \ref{steady_momr}---\ref{steady_momz}) does 
not evolve significantly.  %relative to perturbations 

%Eq. \ref{steady_dust} makes it difficult to obtain equilibrium
%solutions explicitly. However, if the diffusive process is slow and
%can be neglected, then one may just solve
%Eq. \ref{steady_momr}---\ref{steady_momz} with a prescribed (initial)
%distribution of the dust fraction $\tepsilon(r,z)$. 

\subsection{Disk structure with a prescribed dust distribution}\label{steady_state}  
To obtain an actual disk structure for numerical computations, we
assume a Gaussian profile in the dust-to-gas ratio,   
\begin{align}\label{dust_gauss}
  \epsilon(r,z) = \epsilon_0(r)
  \exp{\left[-\frac{z^2}{2\Htilde^2(r)}\right]}. 
\end{align}
%For simplicity we assume the dust-to-gas ratio at the mid-plane
%$\epsilon_0$, as well as its characteristic scale-height
%$\widetilde{H}$, are both constant.
% We take the mid-plane dust-to-gas ratio
%to be a power-law in radius,  
%\begin{align}
%  \epsilon_0(r) = \epsilon_{00}\left(\frac{r}{r_0}\right)^{-d},  
%\end{align}
%where $\epsilon_{00}$ is the dust-to-gas ratio at the fiducial radius
%$r_0$.  
Inserting Eq. \ref{dust_gauss} into vertical hydrostatic equilibrium,
Eq. \ref{steady_momz} and integrating with the approximate
gravitational potential (Eq. \ref{thin_disk_potential}) we obtain the
gas density as
\begin{align}
  &\rhog(r,z)= \notag\\
&\rho_\mathrm{g0}(r)\exp{\left\{ - \frac{z^2}{2\Hgas^2}
    -\epsilon_0\frac{\Htilde^2}{\Hgas^2}\left[1 -
      \exp{\left(-\frac{z^2}{2\Htilde^2}\right)}\right] \right\}}, 
\end{align}
where
\begin{align}
  \Hgas = \frac{c_s}{\OmK}, \quad \OmK \equiv \sqrt{\frac{GM_*}{r^3}},   
\end{align}
is the gas scale-height in the dust-free limit and $\OmK$ is the
Keplerian frequency, respectively. 

We usually consider $\epsilon_0 \ll 1$ and settled dust such that
$\Htilde<\Hgas$. Then $\rhog(r,z)$ is effectively Gaussian, as in the
dust-free case. The dust density is approximately 
\begin{align}
  \rhod \simeq \epsilon_0\rho_\mathrm{g0}(r) \exp
        {\left(-\frac{z^2}{2H_\mathrm{d}^2}\right)}, 
\end{align}
with 
\begin{align}
  \frac{1}{H_\mathrm{d}^2} = \frac{1}{\Htilde^2} + \frac{1}{\Hgas^2}, 
\end{align}
and $H_\mathrm{d}$ is the dust-scale height. In numerical
calculations, we  we specify $H_\mathrm{d}< \Hgas$ to obtain 
$\Htilde$ for input. 

Finally, we define 
\begin{align}
  Z \equiv \epsilon_0\frac{\Hd}{\Hg} \simeq
  \frac{\Sigma_\mathrm{d}}{\Sigma_\mathrm{g}} 
\end{align}
as a measure of the local metalicity, where $\Sigma_\mathrm{d}$ and
$\Sigma_\mathrm{g}$ are the dust and gas surface densities,
respectively. The second equality holds for $\epsilon_0\ll1$.  

%\Sigma_\mathrm{d}$ and the gas
%surface density $\Sigma_\mathrm{g}$.  

\subsection{Orbital frequency} 
In the thin-disk approximation, the disk orbital frequency is 
\begin{align}
  \Omega(r,z) = \OmK(r)\left[1 - \frac{3}{2}\frac{z^2}{r^2} +
    \frac{h_\mathrm{g}^2}{\left(1+\epsilon\right)}\frac{\p}{\p\ln{r}}\ln{\left(c_s^2\rhog\right)}
    \right]^{1/2}, 
\end{align}
where 
\begin{align}
  h_\mathrm{g} \equiv \frac{\Hgas}{r}
\end{align}
is the characteristic disk aspect-ratio. 

\subsection{Vertical shear}
The mixture possess vertical shear. To see this, we eliminate $\Phi$
between Eq. \ref{steady_momr}---\ref{steady_momz} to 
obtain 
\begin{align}\label{vshear}
  r\frac{\p \Omega^2}{\p z} 
%&= \frac{\p\ln{\rho}}{\p r}\frac{\p}{\p
%    z}\left[c_s^2(1-\tepsilon)\right] - \frac{\p\ln{\rho}}{\p z}
%  \frac{\p}{\p r} \left[c_s^2(1-\tepsilon)\right]\\  
   = \frac{1}{\rho}\left(\frac{\p P}{\p r}\frac{\p s}{\p z} -\frac{\p
    P}{\p z}\frac{\p s}{\p r} \right). 
\end{align}
Writing Eq. \ref{vshear} in terms of the gas density and dust-to-gas
ratio with a power-law temperature profile (Eq. \ref{power_temp}) gives 
\begin{align*}
  &r\frac{\p \Omega^2}{\p z}  =
  \frac{c_s^2(r)}{\left(1+\epsilon\right)^2}\left\{
  \frac{\p\epsilon}{\p r}\frac{\p\ln{\rhog}}{\p z}
  -\frac{\p\epsilon}{\p z}\frac{\p\ln{\rhog}}{\p r}\right.\notag\\
  &\phantom{ r\frac{\p \Omega^2}{\p z}  =
    \frac{c_s^2(r)}{\left(1+\epsilon\right)^2}\left\{\right\} }
  \left. -\frac{q}{r}\left[\frac{\p\epsilon}{\p z} + \left(1+\epsilon\right)\frac{\p\ln{\rhog}}{\p z}\right]
  \right\} 
\end{align*}
The first two terms correspond to vertical shear caused by spatial
variations in the dust-to-gas ratio; while the second term
proportional to $q$ corresponds to vertical shear due to the 
radial temperature gradient.

We can compare these sources by
evaluating them using the equilibrium
solutions in \S\ref{steady_state}. We assume the disk is radially
smooth so that $\p_r\sim 1/r$, and the dust-to-gas ratio
$\epsilon\ll1$. This gives 

\begin{align}\label{vshear_split}
  \mathcal{V} =  \frac{\left|r\p_z\Omega\right|_{\text{
        dust/gas gradient}}}{\left|r\p_z\Omega\right|_{\text{
        temp. gradient}}} \sim
  \epsilon \frac{\mathrm{max}\left(\delta^2,
    1\right)}{\left|q\right|\left(\epsilon + \delta^2\right)},
\end{align}
where $\delta\equiv \Htilde/\Hgas$. 
Since $|q|=O(1)$ in PPDs, Eq. \ref{vshear_split} indicates that
vertical shear due to variations in the dust-to-gas ratio can be at
most comparable to that due to the radial temperature gradient; and
this only occurs for thin dust layers with $\delta^2\ll \epsilon$ (so
that $\mathcal{V}\sim 1/|q|$). Otherwise, vertical shear is
predominantly due to $\p_rc_s^2$. 

\subsection{Vertical buoyancy}
Having identified the entropy of an isothermal dusty gas, we find the
vertical buoyancy frequency $N_z$ of the mixture is given by 
\begin{align}
  N_z^2 \equiv - \frac{1}{\rho}\frac{\p P}{\p z}\frac{\p s}{\p z} &=
  \frac{c_s^2(r)}{\left(1+\epsilon\right)^2}\frac{\p\ln\rhog}{\p 
  z}\frac{\p\epsilon}{\p z} \\ &
                                  =
  \frac{\epsilon}{\left(1+\epsilon\right)^2}\left(\frac{z}{\Htilde}\right)^2\OmK^2\notag,  
\end{align}
where the second equality assumes $c_s=c_s(r)$ and the final equality
applies to the equilibria in \S\ref{eqm}. Thus,  
\begin{align*}
N_z\lesssim
O\left(\sqrt{\epsilon_0}\OmK\right). 
\end{align*}
However, for thin dust layers such that $H_\epsilon \ll Hg$, 
$\mathrm{max}\left(N_z\right)$ may occur outside a finite vertical domain.  

We see that $N_z^2\sim \epsilon \OmK^2$, while $r\p_z\Omega^2\sim
h_\mathrm{g}\OmK^2$ for a thin disk where vertical shear is mainly due to the
radial temperature gradient. Since buoyancy is stabilizing, we expect
the dust-induced buoyancy forces to stabilize the disk against the VSI
where $\epsilon \gtrsim h_\mathrm{g}$. 

%\begin{align}
 % N_z^2 
%\end{align}

\section{Limiting behaviours}

In this section we review the basic properties of the 
single-fluid mixture using Eqs.\ref{masseq}--\ref{momeq} with 
Eq. \ref{eff_energy} in place of the dust continuity equation. Our
discussion is based on the corresponding axisymmetric linearized
equations about an axisymmetric steady state. A variable $f$ is
subject to perturbations of the form 
\begin{align}
  \delta f(r,z)\exp{\left(-\ii\sigma t\right)},
\end{align}
where $\sigma$ is the complex mode frequency. In Appendix
\ref{var_prin} we derive the integral relation,  
\begin{align}
&  \sigma^2\int\rho\left(|\dd v_r|^2 + |\dd v_z|^2\right)dV \notag\\
&= \int\left[ \rho
  |\dd v_r|^2A + \rho  \dd v_z \dd v_r^* B + \rho \dd v_z^*\dd v_r C +
  \rho |\dd v_z|^2 D\phantom{\frac{1}{1}}\right. \notag\\
&\phantom{===}  \left. + \frac{1}{P}\left|\nabla\cdot\left(P\dd
  \bm{v}\right)\right|^2\right]dV  -\int \left(\nabla\cdot\dd\bm{v}^*\right)\dd\mathcal{C}dV \notag\\
&\phantom{===}
- \int P
  \left(\nabla\cdot\dd\bm{v}^*\right)\left(\dd\bm{v}\cdot\nabla\ln{c_s^2}\right)dV.\label{int_rel}
\end{align} 
The real coefficients $A,B,C,D$ can be read off 
Eq. \ref{integral_ex}. Note that $B=C$ from dynamical equilibrium
(Eq. \ref{vshear}). We now consider various limits of
Eq. \ref{int_rel}.    


\subsection{Strictly isothermal gas perfectly coupled to dust}   
When $c_s^2$ is a constant and $\tstop=0$, the dusty-gas equations are
exactly equivalent to that for adiabatic hydrodyamics with unit adiabatic
index. Although the gas is strictly isothermal, the mixture behaves 
adiabatically because the dust fraction $\tepsilon$ is advected with 
the gas. This is similar to entropy being conserved following an
adiabatic gas when there is no heating or cooling.  

In this case, the last two integrals in Eq. \ref{int_rel} vanish. Then 
requiring $\sigma^2>0$  gives the criteria 
for the axisymmtric \emph{stability} of an isothermal gas perfectly 
coupled to dust,  
%\begin{align}
%  \kappa^2 + c_s^2 \nabla\ln{\rhog}\cdot\nabla\tepsilon &> 0,\label{dusty_solberg1}  \\
%  -c_s^2\frac{\p\ln{\rhog}}{\p z}\left(-\kappa^2\frac{\p\tepsilon}{\p
%    z} + r\frac{\p\Omega^2}{\p z}\frac{\p \tepsilon}{\p r} \right) & > 0, \label{dusty_solberg2}
%\end{align} 
\begin{align}
  \kappa^2 - \frac{1}{\rho}\nabla P \cdot \nabla s &> 0, \label{dusty_solberg1}    \\
  -\frac{1}{\rho}\frac{\p P}{\p z} \left(\kappa^2 \frac{\p s}{\p z} -
  r\frac{\p\Omega^2}{\p z}\frac{\p s}{\p r}\right) &>0, \label{dusty_solberg2}
\end{align} 
where %$s = \ln{P/\rho}$, 
$\kappa^2 \equiv r^{-3}\p_r\left(r^4\Omega^2\right)$ is the 
square of the epicylic frequency, and we assume $\kappa^2>0$.  
Eq. \ref{dusty_solberg1}---\ref{dusty_solberg2} has the identical form
to the Solberg-Hoiland criteria for adiabatic gas dynamics 
\citep[e.g.][]{tassoul78}, except here the entropy $s$ is really a
representation of the dust distribution via $s=\ln{\left[c_s^2\left(1
    - \tepsilon\right)\right]}$.  

%These criteria are applicable to any adiabatic fluid. 
%For an isothermal mixture, we may take $s=\ln{(1-\tepsilon)}$.   
%from the definition of $s$ and  
%the equation of state (Eq. \ref{eos}) with uniform sound-speed. 
%\begin{align}
%  s \equiv \ln \frac{P}{\rho} = \ln{(1-\tepsilon)} 
%\end{align}
% The first equality is
%similar to the usual defintion for gas entropy (with unit adiabatic
%index). The second follows from the imposed equation of state,
%Eq. \ref{eos}, with a constant sound-speed. 
%The stability conditions are thus
%In accretion disks, $\rhog$, $\rhod$ and $\tepsilon$ are expected to
%decrease away from the mid-plane
%\begin{align*}  
%  \frac{\p\rhog}{\p |z|} < 0,
%, \quad \frac{\p \rhog}{\p r} < 0,   
%\end{align*}
%and similarly for $\rhod$, $\tepsilon$, and the dust-to-gas ratio $\epsilon$. 
%We may thus evaluate the stability criteria at $z>0$ without loss of
%generality. 

%We insert the vertical shear profile,
%Eq. \ref{vshear}, into Eq. \ref{dusty_solberg} and write the result 
%in terms of $\epsilon$, to find 

\subsubsection{Order-of-magnitude estimates} 
We evaluate the above stability criteria for the vertically-Gaussian
distributions for the gas density and the dust-to-gas ratio as 
described in \S\ref{eqm}. 
%However, here we permit 
%$\epsilon_0=\epsilon_0(r)$ and $\Htilde=\Htilde(r)$ to vary with
%radius. 
We assume the disk is approximately Keplerian so that
$\kappa\simeq\OmK$. In terms of the dust-to-gas
ratio $\epsilon$, Eq. \ref{dusty_solberg1}---\ref{dusty_solberg2}
become   
\begin{align}
  &1 + \frac{\epsilon}{\left(1+
    \epsilon\right)^2}\left(h_\mathrm{g}^2\frac{\p\ln{\rhog}}{\p\ln{r}}
  \frac{\p\ln{\epsilon}}{\p\ln{r}} + 
  \frac{z^2}{\Htilde^2}\right)>0 \label{stability_est1},   \\ 
&1 - \frac{\epsilon
  h_\mathrm{g}^2}{\left(1+\epsilon\right)^2}
  \frac{\p\ln{\epsilon}}{\p\ln{r}}\left(-\frac{\p\ln{\rhog}}{\p\ln{r}}+\frac{\Htilde^2}{H_\mathrm{g}^2}\frac{\p\ln{\epsilon}}{\p\ln{r}}\right)
  > 0 \label{stability_est2},
\end{align}
\emph{for stability}. 

Eq. \ref{stability_est1}---\ref{stability_est2} are easily satisfied
in thin accretion disks where radial gradients are $O(1/r)$, 
$h_\mathrm{g},\epsilon,\ll 1$ and $
\Htilde/H_\mathrm{g}\lesssim 1$. Then the
magnitude of the second term on the left-hand-side  of either
inequality is much less than unity. Thus radially smooth, strictly
isothermal disks perfectly coupled with dust are stable against
axisymmetric perturbations.  

%Such disks are stable against
%axisymmetric perturbations.  

\subsubsection{Possible instability at dust edges}
Notice the left-hand-side of Eq. \ref{stability_est2} is a quadratic in
$\p_r\epsilon$. It is possible to
violate this inequality for sufficiently large (in magnitude) radial
gradients  in the dust-to-gas ratio, 
\begin{align}
  \frac{\p\ln{\epsilon}}{\p\ln{r}} > S_+ \quad \mathrm{or} \quad 
  \frac{\p\ln{\epsilon}}{\p\ln{r}} < S_-,
\end{align}
\emph{for instability}, where
\begin{align}\label{spm}
S_\pm = \frac{1}{2}\frac{H_\mathrm{g}^2}{\Htilde^2} 
  \left[
  \frac{\p\ln{\rhog}}{\p\ln{r}} \pm 
  \sqrt{
  \left(\frac{\p\ln{\rhog}}{\p\ln{r}}\right)^2 + 
  4 \frac{\Htilde^2}{H_\mathrm{g}^2}
  \frac{\left(1+\epsilon\right)^2}{\epsilon h_\mathrm{g}^2}
  }
  \,\right]. 
\end{align} 
In typical accretion disks where $\p_r\rhog<0$, it is easier to
achieve instability at a given radius for increasing dust-to-gas
ratios in outwards ($\p_r\epsilon > 0$), and vice versa.  

If the $\p_r\rhog$ can be neglected in Eq. \ref{spm}, then instability
requires 
\begin{align}
  \left|\frac{\p\ln{\epsilon}}{\p r}\right| \gtrsim
  \frac{1}{\Htilde}\frac{\left(1+\epsilon\right)}{\sqrt{\epsilon}}. 
\end{align}
That is, if the dust-to-gas ratio varies on a radial lengthscale less
than $O(\Htilde)$, i.e. its vertical lengthscale,  
%$O(\widetilde{H}\sqrt{\epsilon})$ for $\epsilon\ll 1$, or less than
%$O(\widetilde{H}/\sqrt{\epsilon})$ for $\epsilon\gg 1$, 
then the system is potentially unstable.   



\subsection{Locally isothermal gas perfectly coupled to dust}\label{dusty_vsi_int}
When $c_s(r,z)$ is non-uniform but $\tstop=0$, Eq. \ref{int_rel}
indicates that $\sigma^2$ is generally complex, implying instability.  
Writing $\sigma = 
-\omega + \ii s$ where $\omega$ and $s$ are the real frequency and
growth rates, respectively, we have 
\begin{align}
  s = \frac{\imag\int P
  \left(\nabla\cdot\dd\bm{v}^*\right)\left(\dd\bm{v}\cdot\nabla\ln{c_s^2}\right)dV}{2\omega\int\rho\left(|\dd
    v_r|^2 + |\dd v_z|^2\right)dV}. \label{vsi_check}
\end{align} 
This is classic VSI caused by vertical shear arising from a radial
temperature gradient \citep{nelson13,barker15,lin15}. We compute
explicit solutions in this limit in
\S\ref{linear_problem}---\ref{results}. 

%This is responsible for 
%the usual vertical shear instability in locally isothermal disks.
%However, we may obtain an integral
%relation for linear, axisymmetric waves with frequency $\sigma$ (see
%Appendix \ref{var_prin}, where we also account for finite $\tstop$) in
%the form:   
%If $c_s$ is constant, then $\sigma^2$ is real, and 
%the first integral leads to the previous Solberg-Hoiland criteria for 
%axisymmetric stability. However, when $c_s$ is non-uniform, 

\subsection{Strictly isothermal gas well-coupled to dust} 
Now consider constant $c_s$ but $\tstop\neq0$. Then Eq. \ref{int_rel}
gives 
\begin{align}
  s = \frac{\imag\int \left(\nabla\cdot\dd\bm{v}^*\right)\dd\mathcal{C}dV}{2\omega\int\rho\left(|\dd
    v_r|^2 + |\dd v_z|^2\right)dV}, 
\end{align}
%In this case 
%the effective energy equation is 
%\begin{align}\label{dusty_cooling}
%  \frac{DP}{Dt} &= - P \nabla \cdot \bm{v} + \mathcal{C}
%\end{align} 
%The function $\mathcal{C}$ can be considered as a cooling term. 
%By
%analogy with enabling VSI in an adiabatic gas through cooling
%\citep{nelson13,lin15}, we may expect a `dusty VSI' to develop 
%if this term is large enough (physically meaning diffusion of the
%dust-fraction is strong). 
%Note that $\mathcal{C}$ is also responsible for the streaming
%instability  in the strong-drag limit \citep{laibe14}, though this
%does not require vertical stratification. In the form of
%Eq. \ref{dusty_cooling}, however, the streaming instability has a
%thermodynamical interpretation, similar to thermal instabilities in
%stars \citep{latter06}.  
which represent instabilities due to large-but-finite dust-gas
friction. Our formalism gives these dusty instabilities a 
thermodynamical interpretation because the dust-gas friction enters the problem
as an energy source term \citep[cf. thermal instabilities in
  stars, see also][]{latter06}. 
This includes the streaming instability in
the strong-drag limit \citep{youdin05a, jacquet11, laibe14}. 
