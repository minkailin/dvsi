\section{Single fluid description of well-coupled dusty gas}\label{setup} 
In the limit of strong dust-gas drag (stopping time $\tstop\to 0$) 
\cite{laibe14} give a single fluid description of the mixture:  
(their Eq. 84 --- 85 and Eq. 87): 

\begin{align}
  &\frac{D\rho}{Dt} = -\rho\nabla\cdot\bm{v}, \label{masseq}\\ 
  &\frac{D\bm{v}}{Dt} = - \nabla\Phi - \frac{1}{\rho}\nabla
  P, \label{momeq}\\ 
  &\frac{D\epsilon}{Dt} = -\frac{1}{\rho} \nabla \cdot \left(\epsilon 
  \tstop \nabla P \right),\label{dusteq}  
\end{align}
where $D/Dt \equiv \p_t + \bm{v}\cdot\nabla$ is the Lagragian
derivative; 
\begin{align}
  \rho \equiv \rhog + \rhod
\end{align}
is the total density with $\rhog$ and $\rhod$ being the gas and dust
density, respectively; 
\begin{align}
  \bm{v} \equiv \frac{\rhog\bm{v}_\mathrm{g} + 
    \rhod\bm{v}_\mathrm{d}}{\rho}
\end{align}
is the center-of-mass velocity with $\bm{v}_\mathrm{g}$ and
$\bm{v}_\mathrm{d}$ being the gas and dust velocities, respectively; 
and 
\begin{align}
  \epsilon \equiv \frac{\rhod}{\rho}  = \frac{\tepsilon}{1+\tepsilon} 
\end{align}
is the dust-fraction and $\tepsilon=\rhod/\rhog$ is the usual
dust-to-gas ratio. The gas pressure $P$ is given by the chosen 
equation of state described below. The external potential $\Phi$ is
that for a star of mass $M_*$ at the origin 
\begin{align}
  \Phi(r,z) =-\frac{GM_*}{\sqrt{r^2 + z^2}},
\end{align}
where $G$ is the gravitational constant. 


\subsection{Locally isothermal equation of state for the gas} 
We adopt 
\begin{align}\label{eos}
  P = c_s^2(r,z)\rhog = c_s^{2}(1 - \epsilon)\rho,   
\end{align}
where $c_s$ is a prescribed sound-speed profile fixed in time.  

\subsection{Effective mixture energy equation}
We show that for a prescribed temperature distribution, the mixture
obeys an evolutionary energy equation, due to the advection of the
dust-fraction. 

We eliminate $\epsilon$ in favor of $P$, 
\begin{align}
  \epsilon = 1 - \frac{P}{c_s^2(r,z)\rho}, 
\end{align}
then Eq. \ref{dusteq} becomes

\begin{align}
\frac{DP}{Dt} = - P \nabla\cdot\bm{v} + P\bm{v}\cdot\nabla\ln{c_s^2} +
c_s^2 \nabla\cdot\left[\tstop\left(1 - \frac{P}{c_s^2\rho}\right)\nabla
  P\right]. \label{eff_energy}
\end{align}

This is the energy equation for an ideal gas of adiabatic index
$\gamma=1$, i.e. isothermal evolution, but now with additional source 
terms due to the imposed temperature profile, as well as dust-gas
friction.   

Eq. \ref{eff_energy} can be written in conservative form,
\begin{align*}
  \frac{\p P}{ \p t} + \nabla\cdot\left\{P\left[\bm{v} -
      \tstop\left(c_s^2 - \frac{P}{\rho}\right)\nabla\ln{P}\right]
    \right\}\\
  = \left[P\bm{v} - \tstop\left(c_s^2 - \frac{P}{\rho}\right)\nabla
    P\right]\cdot\nabla\ln{c_s^2}. 
\end{align*}
We may thus re-interpret $P$ as the mixture's energy density, but the
energy flux has an additional contribution from the pressure
gradient. The term on the right-hand-side, owing to the imposed
temperature profile, can be interpreted as an external heat source. 

\section{Equilibrium}
To obtain an axisymmetric steady state with $\rho(r,z)$ and 
$\bm{v}=r\Omega(r,z)\hat{\bm{\phi}}$ where $\Omega = v_\phi/r$, 
we need to solve 
\begin{align}
  r\Omega^2 &= \frac{\p \Phi}{\p r} + \frac{1}{\rho}\frac{\p P}{\p
    r},\label{steady_momr}\\
  0 & = \frac{\p\Phi}{\p z} + \frac{1}{\rho}\frac{\p P}{\p z},\label{steady_momz}\\
  0 & = \nabla\cdot\left(\epsilon\tstop\nabla P\right) \label{steady_dust}
\end{align}
for $\rho$, $\Omega$ and $\epsilon$ with $P=P(\epsilon,\rho)$ given by
the equation of state (Eq. \ref{eos}). 

Eq. \ref{steady_dust} makes it difficult to obtain equilibrium
solutions explicitly. However, if the diffusive process is slow and
can be neglected, then one may just solve
Eq. \ref{steady_momr}---\ref{steady_momz} with a prescribed (initial)
distribution of the dust fraction $\epsilon(r,z)$. 

\subsection{Vertical shear}
The mixture possess vertical shear. To see this, we eliminate $\Phi$
between Eq. \ref{steady_momr}---\ref{steady_momz} to 
obtain 
\begin{align}\label{vshear}
  r\frac{\p \Omega^2}{\p z} &= \frac{\p\ln{\rho}}{\p r}\frac{\p}{\p
    z}\left[c_s^2(1-\epsilon)\right] - \frac{\p\ln{\rho}}{\p z}
  \frac{\p}{\p r} \left[c_s^2(1-\epsilon)\right]\\  
  & = \frac{1}{\rho}\left(\frac{\p P}{\p r}\frac{\p s}{\p z} -\frac{\p
    P}{\p z}\frac{\p s}{\p r} \right),\notag
\end{align}
where
\begin{align}
   s \equiv \ln \frac{P}{\rho}
\end{align}
may be consider as a pseudo-entropy for the mixture. 

\section{Limiting behaviours}

\subsection{Strictly isothermal gas perfectly coupled to dust}  
When $c_s^2$ is a constant and $\tstop=0$, the dusty gas equations are
exactly equivalent to that for adiabatic hydrodyamics with unit adiabatic
index. Although the gas is strictly isothermal, the mixture behaves 
adiabatically because the dust fraction $\epsilon$ is advected with
the gas (i.e. evolves like an entropy when there is no heating or
cooling). 

In this case one may immediately apply the Solberg-Hoiland
criteria to assess axisymmtric \emph{stability}:  

\begin{align}
  \kappa^2 - \frac{1}{\rho}\nabla P \cdot \nabla s &> 0,\\
  -\frac{1}{\rho}\frac{\p P}{\p z} \left(\kappa^2 \frac{\p s}{\p z} -
  r\frac{\p\Omega^2}{\p z}\frac{\p s}{\p r}\right) &>0, 
<<<<<<< HEAD
\end{align}
with $s = \ln{P/\rho^\gamma}$. These criteria are applicable to any adiabatic fluid. 
For the present dusty problem, we may take 
where $\kappa^2 \equiv r^{-3}\p_r\left(r^4\Omega^2\right)$ is the
square of the epicylic frequency. We assume $\kappa^2>0$. 
These criteria are applicable to any adiabatic fluid when $s$
is the usual definition of entropy. For the dusty problem, we may take 
$s=\ln{(1-\epsilon)}+\mathrm{const.}$ from the definition of $s$ and  
the equation of state (Eq. \ref{eos}) with uniform sound-speed. 

%\begin{align}
%  s \equiv \ln \frac{P}{\rho} = \ln{(1-\epsilon)} 
%\end{align}
% The first equality is
%similar to the usual defintion for gas entropy (with unit adiabatic
%index). The second follows from the imposed equation of state,
%Eq. \ref{eos}, with a constant sound-speed. 

The stability conditions are thus
\begin{align}\label{dusty_solberg}
  \kappa^2 + c_s^2 \nabla\ln{\rhog}\cdot\nabla\epsilon &> 0,  \\
  -c_s^2\frac{\p\ln{\rhog}}{\p z}\left(-\kappa^2\frac{\p\epsilon}{\p
    z} + r\frac{\p\Omega^2}{\p z}\frac{\p \epsilon}{\p r} \right) & > 0. 
\end{align}

In accretion disks, $\rhog$, $\rhod$ and $\epsilon$ are expected to
decrease away from the mid-plane
\begin{align*}  
  \frac{\p\rhog}{\p |z|} < 0,
%, \quad \frac{\p \rhog}{\p r} < 0,   
\end{align*}
and similarly for $\rhod$, $\epsilon$, and the dust-to-gas ratio $\tepsilon$. 
We may thus evaluate the stability criteria at $z>0$ without loss of
generality. We insert the vertical shear profile,
Eq. \ref{vshear}, into Eq. \ref{dusty_solberg} and write the result 
in terms of $\tepsilon$, to find 

\begin{align} 
  &1 + \frac{\tepsilon h_\mathrm{g}^2}{\left(1+
    \tepsilon\right)^2}\frac{\Omega^2}{\kappa^2}\left(\frac{\p\ln{\rhog}}{\p\ln{r}}
  \frac{\p\ln{\tepsilon}}{\p\ln{r}} + r^2\frac{\p\ln{\rhog}}{\p z}
  \frac{\p\ln{\tepsilon}}{\p z} \right)>0,   \\
&-\frac{\p\ln{\tepsilon}}{\p z} \notag\\
  &- \frac{\tepsilon
  h_\mathrm{g}^2}{\left(1+\tepsilon\right)^2}\frac{\Omega^2}{\kappa^2}
\frac{\p\ln{\tepsilon}}{\p\ln{r}}\left( \frac{\p\ln{\rhog}}{\p\ln{r}}\frac{\p\ln{\tepsilon}}{\p
    z} -\frac{\p\ln{\rhog}}{\p z}\frac{\p\ln{\tepsilon}}{\p\ln{
    r}}\right) > 0,
%  \kappa^2\left|\frac{\p\epsilon}{\p z}\right| + c_s^2\left|\frac{\p
%    \epsilon}{\p r}\frac{\p\epsilon}{\p z}\frac{\p\ln{\rho}}{\p
%    r}\right| - c_s^2 \left(\frac{\p\epsilon}{\p
%    r}\right)^2\left|\frac{\p\ln{\rho}}{\p z}\right| > 0.  \label{criterion2}
\end{align}
as the stability criteria, where 
\begin{align}
  h_\mathrm{g}\equiv c_s/\left(r\Omega\right)
\end{align}
is the characteristic aspect-ratio of the gas component. 


%This equation shows that the vertical gradient in dust-fraction (and
%thus the dust-to-gas ratio $\rhod/\rhog$) has a stabilizing effect for
%axisymmetric perturbations. Instead, the only way to destabilize the
%system is a sufficiently large radial gradient in the dust-fraction
%and/or vertical (total) density gradient (the last term).  

%Note that Eq. \ref{criterion2} is only applicable for the assume disk
%structure described above: decreasing dust and gas densities, as well
%as the dust-fraction, away from the star and away from the
%midplane. For less typical disk structures, one must evalulate above the
%critera seperately. 



\begin{align} 
  &1 + \frac{\tepsilon h_\mathrm{g}^2}{\left(1+
    \tepsilon\right)^2}\frac{\p\ln{\rhog}}{\p\ln{r}}
  \frac{\p\ln{\tepsilon}}{\p\ln{r}} +
  \tepsilon\frac{z^2}{\widetilde{H}^2}>0,   \\ 
&1 - \frac{\tepsilon
  h_\mathrm{g}^2}{\left(1+\tepsilon\right)^2}
  \frac{\p\ln{\tepsilon}}{\p\ln{r}}\left(-\frac{\p\ln{\rhog}}{\p\ln{r}}+\frac{\widetilde{H}^2}{H_\mathrm{g}^2}\frac{\p\ln{\tepsilon}}{\p\ln{r}}\right) > 0,
\end{align}





\subsubsection{Order-of-magnitude estimates}
Let us assume the gas and dust distributions have a Gaussian
distribution vertically,
\begin{align}
  \rho_\mathrm{g,d}(r,z) = \rho_\mathrm{g0,d0}(r) \exp
      {\left[-\frac{z^2}{2H^2_\mathrm{g,d}(r)}\right]}.
\end{align}
Then the dust-to-gas ratio is 
\begin{align}
  \tepsilon = \tepsilon_0(r)
  \exp{\left[-\frac{z^2}{2\widetilde{H}^2(r)}\right]}, 
\end{align}
with $\tepsilon_0= \rho_\mathrm{d,0}/\rho_{g_0}$, and  
\begin{align}
  \frac{1}{\widetilde{H}^2} \equiv \frac{1}{H_\mathrm{d}^2} -
  \frac{1}{H_\mathrm{g}^2}. 
\end{align}
%We assume the dust layer is much thinner than the gas scale-height,
%$H_\mathrm{d}\ll H_\mathrm{g}$, so that 
%\begin{align}
% \widetilde{H} \sim H_\mathrm{d}. 
%\end{align} 

We further assume
%the gas and dust
%distributions are smoothly varying in the radial direction so that  
%\begin{align}
%  \frac{\p}{\p r}\sim \frac{p}{r}, 
%\end{align}
the disk is roughly Keplerian so that
\begin{align}
  \kappa \sim \OmK, 
\end{align}
and the gas sound-speed is 
\begin{align}
  c_s \sim h_\mathrm{g}r\OmK, 
\end{align}
where $h_\mathrm{g}\ll1$ is the aspect-ratio of the gas component. 

Inserting the above approximations into Eq. \ref{criterion2} and writing the result in terms of the
dust-to-gas ratio $\tepsilon$,           
we find 
\begin{align}\label{ineq_est}
  1 + \frac{\tepsilon h_\mathrm{g}^2}{\left(1 + \tepsilon\right)^2}
  \left|\frac{\p\ln{\tepsilon}}{\p\ln{r}}\right| 
  \left[
    \left|\frac{\p\ln{\rhog}}{\p\ln{r}}\right| - \frac{\widetilde{H}^2}{H^2_\mathrm{g}}\left|\frac{\p\ln{\tepsilon}}{\p\ln{r}}\right|  
    \right]>0
\end{align}
is required \emph{for stability.} 
This inequality is easily satisifed in typical disk models where 
the dust layer is much thinner than the gas scale-height, so
that $\widetilde{H}\sim H_\mathrm{d}\ll H_\mathrm{g}$, and where the radial profiles smoothly vary on a scale of $r$.
Then the magnitude the only destabilizing term (the third term) is negligible. According to Eq. \ref{ineq_est}, instability is only possible if there is a rapid variation the radial profile of the dust-to-gas ratio.   


\subsection{Locally isothermal gas perfectly coupled to dust} 
When $c_s(r,z)$ is non-uniform but $\tstop=0$, simple stability
criteria are not available. However, we may obtain an integral
relation for linear, axisymmetric waves with frequency $\sigma$ (see
Appendix \ref{var_prin}) in the form:  
\begin{align}
&  \sigma^2\int\rho\left(|\dd v_r|^2 + |\dd v_z|^2\right)dV \notag\\
&= \int\left[ \rho
  |\dd v_r|^2A + \rho  \dd v_z \dd v_r^* B + \rho \dd v_z^*\dd v_r C +
  \rho |\dd v_z|^2 D\phantom{\frac{1}{1}}\right. \notag\\
&\phantom{===}  \left. + \frac{1}{P}\left|\nabla\cdot\left(P\dd
  \bm{v}\right)\right|^2\right]dV - \int P
  \left(\nabla\cdot\dd\bm{v}^*\right)\left(\dd\bm{v}\cdot\nabla\ln{c_s^2}\right)dV.
\end{align}
The real coefficients $A,B,C,D$ can be read off
Eq. \ref{integral_ex}. Note that $B=C$ from the assume equilibrium
structure (Eq. \ref{vshear}).  

If $c_s$ is constant, then $\sigma^2$ is real, and 
the first integral leads to the usual Solberg-Hoiland criteria for
axisymmetric stability. However, when $c_s$ is non-uniform, $\sigma^2$
is generally complex, implying instability. This is responsible for
the usual vertical shear instability in locally isothermal disks. 

\subsection{Strictly isothermal gas well-coupled to dust} 
Now consider $c_s^2=\mathrm{const.}$ but $\tstop\neq0$. In this case
the effective energy equation is 
\begin{align}\label{dusty_cooling}
  \frac{DP}{Dt} &= - P \nabla \cdot \bm{v} + \mathcal{C},\\
  \mathcal{C} & = \nabla\cdot\left[\tstop\left(c_s^2 -
    \frac{P}{\rho}\right)\nabla P\right].           
\end{align} 
The function $\mathcal{C}$ can be considered as a cooling term. By
analogy with enabling VSI in an adiabatic gas through cooling
\citep{nelson13,lin15}, we may expect a `dusty VSI' to develop 
if this term is large enough (physically meaning diffusion of the
dust-fraction is strong). 


Note that $\mathcal{C}$ is also responsible for the streaming
instability  in the strong-drag limit \citep{laibe14}, though this
does not require vertical stratification. In the form of
Eq. \ref{dusty_cooling}, however, the streaming instability has a
thermodynamical interpretation, similar to thermal instabilities in
stars \citep{latter06}.  
