\section{Linear problem for perfectly coupled dust}\label{linear_problem}
In this paper we calculate explicit solutions to the linearized
equations in the limit of perfectly coupled dust ($\tstop=0$), so that
the equilibria defined in \S\ref{eqm} are exact steady states; and we
consider radially-localized axisymmetric disturbances of the form  
\begin{align}
  \delta f (r, z) = \delta f_1(r,z)\exp{(\ii k r)},
\end{align} 
%and similarly for $\dd P$ and $\dd\bm{v}$. 
where $k$ is a real wavenumber such that $|kr|\gg 1$, and the
amplitude $\dd f_1(r,z)$ is 
a slowly-varying function of $r$. Then 
$\p_r\to i k$ when acting on the above primitive perturbations, and we may
neglect curvature terms. We take  
$k>0$ without loss of generality. Hereafter, we drop the subscript 1
on the amplitudes. 
%The frequency $\sigma = \omega +
%\ii s$ is generally complex, with $\omega$ being the real frequency
%and $s$ is the real growth rate. 

Introducing 
\begin{align}
  W \equiv \frac{\dd\rho}{\rho}, \quad Q \equiv \frac{\dd P}{\rho},
\end{align}
the linearized equations for 
locally isothermal, perfectly-coupled dusty gas with the pressure
equation in place of the dust-fraction
(Eq. \ref{masseq}---\ref{momeq}, Eq. \ref{eff_energy}) are then:    

\begin{align}
  \ii\sigma W &= \ii k \dd v_r + \dd v_z^\prime +
  \dd v_r \p_r\ln{\rho} + \dd v_z\p_z\ln{\rho},\label{lin_mass}\\
  -\ii\sigma\dd v_r  &= 2\Omega\dd v_\phi + 
  \delta\bm{F}\cdot\hat{\bm{r}},\label{lin_xmom}\\
  \ii\sigma\dd v_\phi &= \frac{\kappa^2}{2\Omega}\dd v_r + \frac{\p
    v_\phi}{\p z}\dd v_z, \label{lin_ymom}\\
  -\ii\sigma\dd v_z &=  \delta\bm{F}\cdot\hat{\bm{z}},\label{lin_zmom}\\
  \ii\sigma Q &= \frac{P}{\rho}\left(\ii k \dd v_r + \dd
               v_z^\prime\right) + \frac{1}{\rho}\left(\dd v_r\p_rP + \dd v_z \p_zP\right)\notag\\
                &\phantom{=}-\frac{P}{\rho} \dd v_r\p_r
               \ln{c_s^2}, \label{lin_energy} 
%+ \dd v_z \p_z\ln{c_s^2}\right),\label{lin_energy} 
%               - \frac{\dd\mathcal{C}}{\rho},\label{lin_energy} 
\end{align}  
where $^\prime \equiv \p_z$ and $\dd\bm{F}$ is the linearized pressure
force, given in Appendix \ref{lin_press}. Note that we have assumed a
temperature profile that only depends on $r$.  

%; and 
%\begin{align}
%  \delta \bm{F} \equiv \frac{\dd\rho}{\rho^2}\nabla P -
%  \frac{1}{\rho}\nabla\dd P, 
%\end{align}
%$\dd\mathcal{C}$ is the linearized dust-duffusion function, given in
%Appendix \ref{lin_dust}. We consider stopping times appropriate for
%small grains in the Epstein regime. Note that for the axisymmetric
%problem, $\dd\bm{F}$ is purely meridional. 

Eq. \ref{lin_mass}---\ref{lin_energy} is a set of ordinary
differntial equations in $z$. All coefficients and amplitudes are
evaluated at a fiducial radius $r=r_0$, but their full $z$-dependence
is retained.  Two boundary conditions are needed for the $\tstop=0$
problem. This is expected since this problem is exactly equivalent to
that in adiabatic hydrodynamics \citep[e.g.][]{lubow93}. For simplicity
we impose solid boundaries so that $\delta v_z(\pm\zmax)=0$ and
$\zmax=5\Hgas$. 

We solve the linearized equations as a generalized eigenvalue problem 
using a pseudo-spectral code adapted from \cite{lin15}. Amplitudes 
are expanded in Chebyshev polynomials up to order $N_z-1$. Our 
standard resolution is $N_z=513$. We check results using
Eq. \ref{vsi_check}.    

%\subsection{Boundary conditions}



























%Eq. \ref{lin_mass}---\ref{lin_energy} can be reduced to a set of
%first-order differential equations for $W, Q$ and $\dd v_z$. We can
%see this schematically as follows. Eq. \ref{lin_xmom} and
%\ref{lin_ymom} may be combined to yield 
%\begin{align*}
%  \dd v_r = \dd v_r (W, Q, \dd v_z). 
%\end{align*} 
%We can then take the continuity equation as an equation for $\dd v_z$, 
%\begin{align*}
%\text{Eq. \ref{lin_mass}} \Rightarrow \dd v_z^\prime(z) = \dd
%  v_z^\prime(W,Q,\dd v_z),
%\end{align*}
%and the vertical momentum equation as an equation for $Q$, 
%\begin{align*}
%\text{Eq. \ref{lin_zmom}} \Rightarrow Q^\prime(z) = 
% Q^\prime(W,Q,\dd v_z). 
%\end{align*}

%Now, for finite dust-gas coupling, $\tstop\neq0$, inspection of $\dd C$
%(Appendix \ref{lin_dust}) shows that it involves $W^\prime$ (assuming
%$\epsilon < 1$). Then Eq. \ref{lin_energy} may be taken as a
%differential equation for $W$. However, for perfectly coupled dust,
%$\tstop =\dd\mathcal{C}= 0$. In that case Eq. \ref{lin_energy} gives an algebraic
%relation between $W$, $Q$ and $\dd v_z$. That is,
%\begin{align*}
%\text{Eq. \ref{lin_energy}}\Rightarrow \begin{cases}
%  W^\prime(z) = W^\prime(W, Q, \dd v_z) & \text{if } \tstop \neq 0, \\
%  W           (z) = W(Q, \dd v_z) & \text{if }\tstop = 0.
%\end{cases}
%\end{align*} 

%This means that for the perfectly coupled problem, we have a pair of
%ODEs for $Q$ and $\dd v_z$, and 
%When
%$\tstop\neq0$, we require three boundary conditions. {\bf WHAT?? Why odd
%  number of BCs? Shouldn't we have even number of BCs? We have
%two boundaries? I've only seen problems like this involving even
%number of BCs.} In this case we impose $\dd v_z =0$ at $z=\pm
%z_\mathrm{max}$, and classify modes according to their symmetry about
%the midplane: `even' modes with $\dd v_z^\prime(0) = 0$, and `odd' 
%modes with $\dd v_z(0)=0$.       

