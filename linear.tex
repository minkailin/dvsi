\section{Linearized equations}\label{linear_problem}
To compute explicit solutions to the linear problem, 
we consider radially-localized axisymmetric disturbances of the form  
\begin{align}
  \delta f (r, z) = \delta f_1(r,z)\exp{(\ii k_x r)},
\end{align} 
%and similarly for $\dd P$ and $\dd\bm{v}$. 
where $k_x$ is a real wavenumber such that $|k_xr|\gg 1$, and the 
amplitude $\dd f_1(r,z)$ is 
a slowly-varying function of $r$. Then 
$\p_r\to i k_x$ when acting on the above primitive perturbations, and we may
neglect curvature terms. We take  
$k_x>0$ without loss of generality. Hereafter, we drop the subscript 1
on the amplitudes. 
%The frequency $\sigma = \omega +
%\ii s$ is generally complex, with $\omega$ being the real frequency
%and $s$ is the real growth rate. 

Introducing 
\begin{align}
  W \equiv \frac{\dd\rho}{\rho}, \quad Q \equiv \frac{\dd P}{\rho},
\end{align}
the linearized equations for 
locally isothermal dusty gas with the pressure
equation in place of the dust-fraction
(Eq. \ref{masseq}---\ref{momeq}, Eq. \ref{eff_energy}) are then:    

\begin{align}
  \ii\sigma W &= \ii k_x \dd v_r + \dd v_z^\prime +
  \dd v_r \p_r\ln{\rho} + \dd v_z\p_z\ln{\rho},\label{lin_mass}\\
  -\ii\sigma\dd v_r  &= 2\Omega\dd v_\phi 
% +  \delta\bm{F}\cdot\hat{\bm{r}}
- W F_r - \ii k_x Q,\label{lin_xmom}\\
  \ii\sigma\dd v_\phi &= \frac{\kappa^2}{2\Omega}\dd v_r + \frac{\p
    v_\phi}{\p z}\dd v_z, \label{lin_ymom}\\
  -\ii\sigma\dd v_z &= - W F_z - \left[Q^\prime + Q
    \left(\ln{\rho}\right)^\prime\right]  %\delta\bm{F}\cdot\hat{\bm{z}} 
,\label{lin_zmom}\\
  \ii\sigma Q &= \frac{P}{\rho}\left(\ii k_x \dd v_r + \dd
               v_z^\prime\right) + \frac{1}{\rho}\left(\dd v_r\p_rP + \dd v_z \p_zP\right)\notag\\
                &\phantom{=}-\frac{P}{\rho} \dd v_r\p_r
               \ln{c_s^2} %, \label{lin_energy} 
%+ \dd v_z \p_z\ln{c_s^2}\right),\label{lin_energy} 
               - \frac{\dd\mathcal{C}}{\rho},\label{lin_energy} 
\end{align}  
where $^\prime \equiv \p_z$ and recall $\bm{F} \equiv -\nabla P/\rho$. 
The linearized dust-duffusion function 
$\dd\mathcal{C}$ is given in  Appendix \ref{lin_dust}. 
% and $\dd\bm{F}$ is the linearized pressure
%force, given in Appendix \ref{lin_press}. 
Note that we have assumed a temperature profile that only depends on $r$.   

%; and 
%\begin{align}
%  \delta \bm{F} \equiv \frac{\dd\rho}{\rho^2}\nabla P -
%  \frac{1}{\rho}\nabla\dd P, 
%\end{align}
%$\dd\mathcal{C}$ is the linearized dust-duffusion function, given in
%Appendix \ref{lin_dust}. We consider stopping times appropriate for
%small grains in the Epstein regime. Note that for the axisymmetric
%problem, $\dd\bm{F}$ is purely meridional. 

Eq. \ref{lin_mass}---\ref{lin_energy} is a set of ordinary
differntial equations in $z$. All coefficients and amplitudes are
evaluated at a fiducial radius $r=r_0$, but their full $z$-dependence
is retained. We now discuss explicit solutions to these equations. 
% Numerical solutions are generally required for the 
%vertically-stratified problem. 

\section{Compressible streaming instability}\label{si}
In the special case of an unstratified disk (neglecting the vertical
component of the stellar gravity), we may also Fourier analyze in $z$
to obtain algebraic dispersion relation of the form  
$\sum_{j=0}^{5}c_j(k_x,k_z)\sigma^j = 0$, where $k_z$ is a real
vertical wavenumber. The coefficients $c_j$ can be read 
off Eq. \ref{streaming_dispersion} in  Appendix \ref{compressible_streaming}. 
There we also show that this dispersion relation reduces to that for
the streaming instability in the limit of incompressible gas and small
$\tstop$ \citep{youdin05a,jacquet11}.   

{\bf note: kappa2 accounts for dust effect}
We solve Eq. \ref{streaming_dispersion} for selected cases where the  
growth rates of the streaming instability have been 
calculated analytically from full two-fluid disk models and measured 
from direct numerical particle-gas simulations 
\citep[namely][]{youdin07b,bai10b}. These authors use normalized
wavenumbers $K_x = \eta r k_x$ and similarly for $K_z$, where
\begin{align} 
  \eta \equiv -\frac{1}{2\rhog r\OmK^2}\frac{\p P}{\p r} = 
  \frac{1}{2\left(1-\tepsilon\right)}\frac{F_r}{r\OmK^2}, 
\end{align} 
measures the pressure offset of Keplerian rotation. We use
$\eta=0.05c_s/r\OmK$. They also define the stopping time as 
$\tau_\mathrm{s}=\tstop/(1-\tepsilon)$. 
 
Table \ref{si_compare} compares the eigenfrequencies obtained from the
one-fluid dispersion relation to that from the above studies. As  
expected eigenfrequencies agree better with decreasing $\tstop$ since
in that limit the mixture behaves more like a single fluid. Most 
importantly, we find the work done $\mathcal{W}>0$ in all cases, and
hence find growing oscillations. Thus the streaming instability stems
from the pressure evolution lagging behind that of the total density.  

Notice also there is better agreement between eigenfrequencies for
smaller $\mathcal{W}$, i.e. smaller phase pressure-density phase
lags. While this is generally achieved with decreasing $\tstop$, other
problem parameters may affect the magnitude of this phase lag.
For example, `linA' and `linB' have the same $\tstop$ but linA shows
better agreement with the one-fluid growth rate. This suggest that how
well dust and gas are effectively coupled is problem-dependent. 

%"generalized stokes number" would include other problem parameters,
%e.g. wavenumber  

\begin{deluxetable*}{llrrrrr}
  \tablecolumns{7}
  \tablecaption{Selected eigenfrequencies for the streaming
    instability. \label{si_compare}
  }
  \tablehead{
    \colhead{Mode} & 
    \colhead{$\tau_\mathrm{s}\OmK$} &
    \colhead{$\epsilon$} &
     \colhead{$K_x = K_z$} &
      \colhead{$\sigma/\OmK$ (two-fluid)} &
    \colhead{$\sigma/\OmK$ (one-fluid, Eq. \ref{streaming_dispersion})}   &
    \colhead{$\mathcal{W}$ (arbitrary units) } 
  }
\startdata
 linA, \cite{youdin07b} &  $0.1$       & 3.0 & 30    & $-0.3480 +
 0.4190\ii$ & $0.3508 + 0.4195\ii$ & $0.026$\\ 

linB, \cite{youdin07b} & $0.1$        &  0.2 & 6 & $0.4999 +
0.0155\ii$&   $-0.4570 + 0.0038\ii$  & $2.00$ \\

linC,  \cite{bai10b}  & $10^{-2}$   &  2.0 & 1500&   $0.1049 +
0.5981\ii$   &  $0.1221 + 0.6356\ii$  & $3.27\times10^{-4}$\\

linD, \cite{bai10b} &  $10^{-3}$   &  2.0 & 2000 & $0.3225 +
0.3154\ii$& $0.3040 + 0.3123\ii$ &  $2.00\times10^{-4}$
\enddata
\end{deluxetable*}



























%\subsection{Boundary conditions}
%Eq. \ref{lin_mass}---\ref{lin_energy} can be reduced to a set of
%first-order differential equations for $W, Q$ and $\dd v_z$. We can
%see this schematically as follows. Eq. \ref{lin_xmom} and
%\ref{lin_ymom} may be combined to yield 
%\begin{align*}
%  \dd v_r = \dd v_r (W, Q, \dd v_z). 
%\end{align*} 
%We can then take the continuity equation as an equation for $\dd v_z$, 
%\begin{align*}
%\text{Eq. \ref{lin_mass}} \Rightarrow \dd v_z^\prime(z) = \dd
%  v_z^\prime(W,Q,\dd v_z),
%\end{align*}
%and the vertical momentum equation as an equation for $Q$, 
%\begin{align*}
%\text{Eq. \ref{lin_zmom}} \Rightarrow Q^\prime(z) = 
% Q^\prime(W,Q,\dd v_z). 
%\end{align*}

%Now, for finite dust-gas coupling, $\tstop\neq0$, inspection of $\dd C$
%(Appendix \ref{lin_dust}) shows that it involves $W^\prime$ (assuming
%$\tepsilon < 1$). Then Eq. \ref{lin_energy} may be taken as a
%differential equation for $W$. However, for perfectly coupled dust,
%$\tstop =\dd\mathcal{C}= 0$. In that case Eq. \ref{lin_energy} gives an algebraic
%relation between $W$, $Q$ and $\dd v_z$. That is,
%\begin{align*}
%\text{Eq. \ref{lin_energy}}\Rightarrow \begin{cases}
%  W^\prime(z) = W^\prime(W, Q, \dd v_z) & \text{if } \tstop \neq 0, \\
%  W           (z) = W(Q, \dd v_z) & \text{if }\tstop = 0.
%\end{cases}
%\end{align*} 

%This means that for the perfectly coupled problem, we have a pair of
%ODEs for $Q$ and $\dd v_z$, and 
%When
%$\tstop\neq0$, we require three boundary conditions. {\bf WHAT?? Why odd
%  number of BCs? Shouldn't we have even number of BCs? We have
%two boundaries? I've only seen problems like this involving even
%number of BCs.} In this case we impose $\dd v_z =0$ at $z=\pm
%z_\mathrm{max}$, and classify modes according to their symmetry about
%the midplane: `even' modes with $\dd v_z^\prime(0) = 0$, and `odd' 
%modes with $\dd v_z(0)=0$.       

