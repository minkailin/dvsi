\section{Linearized equations}\label{linear_problem}
To compute explicit solutions to the linear problem, 
we consider radially-localized axisymmetric disturbances of the form  
\begin{align}
  \delta X (r, z) = \delta X_1(r,z)\exp{(\ii k_x r)},
\end{align} 
%and similarly for $\dd P$ and $\dd\bm{v}$. 
where $k_x$ is a real wavenumber such that $|k_xr|\gg 1$, and the 
amplitude $\dd X_1(r,z)$ is 
a slowly-varying function of $r$. Then 
$\p_r\to i k_x$ when acting on the above primitive perturbations, and we may
neglect curvature terms. We take  
$k_x>0$ without loss of generality. Hereafter, we drop the subscript 1
on the amplitudes. 
%The frequency $\sigma = \omega +
%\ii s$ is generally complex, with $\omega$ being the real frequency
%and $s$ is the real growth rate. 

Introducing 
\begin{align}
  W \equiv \frac{\dd\rho}{\rho}, \quad Q \equiv \frac{\dd P}{\rho},
\end{align}
the linearized equations for 
locally isothermal dusty gas with the pressure
equation in place of the dust-fraction
(Eq. \ref{masseq}---\ref{momeq}, Eq. \ref{eff_energy}) are then:    

\begin{align}
  \ii\sigma W &= \ii k_x \dd v_r + \dd v_z^\prime +
  \dd v_r \p_r\ln{\rho} + \dd v_z\p_z\ln{\rho},\label{lin_mass}\\
  -\ii\sigma\dd v_r  &= 2\Omega\dd v_\phi 
% +  \delta\bm{F}\cdot\hat{\bm{r}}
- W F_r - \ii k_x Q - \ii k_x\dd\psi,\label{lin_xmom}\\
  \ii\sigma\dd v_\phi &= \frac{\kappa^2}{2\Omega}\dd v_r + \frac{\p
    v_\phi}{\p z}\dd v_z, \label{lin_ymom}\\
  -\ii\sigma\dd v_z &= - W F_z - \left[Q^\prime + Q
    \left(\ln{\rho}\right)^\prime\right] - \dd\psi^\prime  %\delta\bm{F}\cdot\hat{\bm{z}} 
,\label{lin_zmom}\\
  \ii\sigma Q &= \frac{P}{\rho}\left(\ii k_x \dd v_r + \dd
               v_z^\prime\right) + \frac{1}{\rho}\left(\dd v_r\p_rP + \dd v_z \p_zP\right)\notag\\
                &\phantom{=}-\frac{P}{\rho} \dd v_r\p_r
               \ln{c_s^2} %, \label{lin_energy} 
%+ \dd v_z \p_z\ln{c_s^2}\right),\label{lin_energy} 
               - \frac{\dd\mathcal{C}}{\rho},\label{lin_energy}\\
\dd\psi^{\prime\prime}  &= 4\pi G \rho W + k_x^2\dd\psi, \label{lin_sg}
\end{align}  
where $^\prime \equiv \p_z$ and recall $\bm{F} \equiv -\nabla P/\rho$. 
The linearized dust-duffusion function 
$\dd\mathcal{C}$ is given in  Appendix \ref{lin_dust}. 
% and $\dd\bm{F}$ is the linearized pressure
%force, given in Appendix \ref{lin_press}. 
Note that we have assumed a temperature profile that only depends on $r$.   

%; and 
%\begin{align}
%  \delta \bm{F} \equiv \frac{\dd\rho}{\rho^2}\nabla P -
%  \frac{1}{\rho}\nabla\dd P, 
%\end{align}
%$\dd\mathcal{C}$ is the linearized dust-duffusion function, given in
%Appendix \ref{lin_dust}. We consider stopping times appropriate for
%small grains in the Epstein regime. Note that for the axisymmetric
%problem, $\dd\bm{F}$ is purely meridional. 

Eq. \ref{lin_mass}---\ref{lin_sg} is a set of ordinary
differntial equations in $z$. All coefficients and amplitudes are
evaluated at a fiducial radius $r=r_0$, but their full $z$-dependence
is retained. We now discuss solutions to these equations for a
razor-thin, self-gravitating disk in \S\ref{sgi}; and for 3D
non-self-gravitating, unstratified and stratified disks 
in \S\ref{si} and \S\ref{results}, respectively.   
 
 % Numerical solutions are generally required for the 
%vertically-stratified problem. 


\section{Secular gravitational instability}\label{sgi}
We first consider a razor-thin disk so that $\rho =
\Sigma\delta(z)$. Here $\delta$ is the delta function and $\Sigma$ is
the total surface density. Note that $\epsilon$ and 
$\tepsilon$ now refers to the global dust-to-gas ratio and dust 
fraction, respectively. We neglect the vertical dimension   
in Eq. \ref{lin_mass}---\ref{lin_energy}. The background disk is
uniform. The linearized dust-gas drag term is then
$-\delta\mathcal{C}/\rho = \tstop\tepsilon c_s^2k_x^2 Q$. The
thin-disk solution to Eq. \ref{lin_sg} is $\dd\psi = -2\pi G
\Sigma W/\left|k_x\right|$    

These simplifications yield the dispersion relation
\begin{align*}
  \left(\ii\sigma - \tstop\tepsilon c_s^2k_x^2\right)\left( 2 \pi G
    \Sigma \left|k_x\right|  - \kappa^2 + \sigma^2 \right) = \ii
  \sigma c_s^2 k_x^2\left(1 - \tepsilon\right). 
\end{align*}
Searching for slowly and purely growing modes, $\sigma = \ii s$ with 
$|s|\ll \kappa$, we find 
\begin{align}  
s = \frac{\tstop\tepsilon c_s^2 k_x^2 \left( 2 \pi \Sigma G
    \left|k_x\right| - \kappa^2\right)}{\kappa^2 -2 \pi \Sigma G
    \left|k_x\right| + c_s^2k_x^2\left(1-\tepsilon\right) }. \label{sgi_disp}
\end{align}
This is just the secular gravitational instability in the absence of
turbulent dust diffusion but strong drag
\citep[][ their Eq. 13 becomes Eq. \ref{sgi_disp} in this limit with
a change of variables]{takahashi14}. This is gravitational instability
mediated by dust-gas drag. A similar effect occurs in viscous
self-gravitating gas disks \citep{gammie96,lin16}. In fact, if we
identify $\nu \equiv \tstop c_s^2$ as a kinematic viscosity, then
Eq. \ref{sgi_disp} is effectively 
\citeauthor{gammie96}'s Eq. 18. 

This exercise shows that the one-fluid framework, futher simplified by
the terminal velocity approximation, is sufficient to capture the SGI
in the strong drag limit. 

\section{Streaming instability}\label{si}
We now consider 3D but non-self-gravitating disks.  
In the special case of an unstratified disk (neglecting the vertical
component of the stellar gravity), we may also Fourier analyze in $z$
to obtain an algebraic dispersion relation of the form  
$\sum_{j=0}^{5}c_j(k_x,k_z)\sigma^j = 0$, where $k_z$ is a real
vertical wavenumber. The coefficients $c_j$ can be read 
off Eq. \ref{streaming_dispersion} in  Appendix \ref{compressible_streaming}. 
There we also show that this dispersion relation reduces to that for
the streaming instability (SI) in the limit of incompressible gas and small
$\tstop$ \citep{youdin05a,jacquet11}.   
 
%\subsection{Numerical examples}

%{\bf note: kappa2 accounts for dust effect assuming smallh=0.05}

We solve the full dispersion relation, Eq. \ref{streaming_dispersion} for selected cases where SI  
growth rates have been 
calculated analytically from full two-fluid disk models and confirmed
with particle-gas numerical simulations 
\citep[namely][]{youdin07b,bai10b}. These authors use normalized
wavenumbers $K_{x,z} = \eta r k_{x,z}$ where
\begin{align} 
  \eta \equiv -\frac{1}{2\rhog r\OmK^2}\frac{\p P}{\p r} = 
  \frac{1}{2\left(1-\tepsilon\right)}\frac{F_r}{r\OmK^2}, 
\end{align} 
measures the pressure offset of Keplerian rotation. We use
$\eta=0.05c_s/r\OmK$. They also define the stopping time as
$\tau_\mathrm{s}=\tstop/(1-\tepsilon)$.  

Table \ref{si_compare} compares the eigenfrequencies obtained from the
one-fluid dispersion relation to that from the above studies. As  
expected eigenfrequencies agree better with decreasing $\tstop$ since
in that limit the mixture behaves more like a single fluid. Most 
importantly, we find the work done $\mathcal{W}>0$ in all cases, and
hence find growing oscillations. 

%{\bf note: important to use largrangian pressure pert properly}

We also checked the growth rates $s$ satisfy 
\begin{align} 
  s = \frac{\left|\sigma\right|^2\imag\left(\Delta P
    \Delta\rho^*/\rho\right)}{2\real\left(\sigma\right)\rho\left(\left|\dd
  v_x\right|^2+\left|\dd
  v_z\right|^2\right)}, \label{si_check}
\end{align}
as implied by Eq. \ref{thermal_instability} and
Eq. \ref{pdv}. %, or directly from
%Eq. \ref{streaming_mass}---\ref{streaming_vz}. 
Interestingly, we find that for cases with
$\epsilon>1$, Eq. \ref{si_check} can be satisified with $\Delta
P\simeq \ii\dd v_x \p_rP/\sigma$, i.e. the radial pressure gradient
is responsible for growth. Conversely, for `linB' with $\epsilon
< 1$, one can approximate $\Delta P \simeq \dd P$ in Eq. \ref{si_check}. This
suggests that for SI in gas-dominated disks, the distinction between
Eulerian and Lagragian pressure perturbations is unimportant. 
%{\bf but weak growth in this case}
%However
%in that case the gro 

In Table \ref{si_compare} we also calculate the phase difference
between the Lagragian pressure perturbation and density perturbations
as    
\begin{align*} 
\varphi \equiv -\sgn(\omega)\arg\left(\Delta P\Delta\rho^*\right)
        = \sgn(\omega)\arg\left(\Delta\rhog^*\Delta\rhod\right),\notag
\end{align*}
%where we have applied the isothermal equation of state for the second equality.
then $\varphi > 0 $ indicates pressure lagging behind density, which
is true for all the cases. In particular, `linB' indicates that growth
rates vanish as $\varphi \to 0$.  
Thus the streaming instability indeed stems  
from the gas pressure evolution lagging behind that of the dust 
density. 

%"generalized stokes number" would include other problem parameters,
%e.g. wavenumber  


%{\bf suppose large scale pressure grad drops outwards. 
%pert the sys by kicking more dust inwards. gas moves out, into region 
%  of lower pressure in the bg. i.e. create pressure bump at larger
%  radius. it attracts the first dust particles back out, plus some more
%  because bump is larger. (move gas from higher density to lower
%  density produce asymmetric bump/trough.}

\begin{deluxetable*}{llrrrrrr}
  \tablecolumns{8}
  \tablecaption{Selected eigenfrequencies for the streaming
    instability. \label{si_compare}
  }
  \tablehead{
    \colhead{Mode} & 
    \colhead{$\tau_\mathrm{s}\OmK$} &
    \colhead{$\epsilon$} &
     \colhead{$K_{x,z}$} &
      \colhead{$\sigma/\OmK$ (two-fluid)} &
    \colhead{$\sigma/\OmK$ (one-fluid)} &
    \colhead{$\mathcal{W}$ (arbitrary units)} &
      \colhead{$\Delta P$ lag}
  }
\startdata
 linA, \cite{youdin07b} &  $0.1$       & 3.0 & 30    & $-0.3480 +
 0.4190\ii$ & $0.3640 + 0.4249\ii$ & $0.90$ & $30\degr$\\ 

linB, \cite{youdin07b} & $0.1$        &  0.2 & 6 & $0.4999 +
0.0155\ii$&   $-0.4981 + 0.0054\ii$  & $1.54$ & $1.2\degr$ \\

linC,  \cite{bai10b}  & $10^{-2}$   &  2.0 & 1500&   $0.1049 +
0.5981\ii$   &  $0.1338 + 0.6650\ii$  & $0.15$& $11\degr$ \\

linD, \cite{bai10b} &  $10^{-3}$   &  2.0 & 2000 & $0.3225 + 
0.3154\ii$& $0.3219 + 0.3154\ii$ &  $1.28$ & $22\degr$ 
\enddata
\end{deluxetable*}


% \subsection{Eulerian interpretation}

% We can also consider SI in the Eulerian sense \citep[as
% did][]{jacquet11}. Eq. \ref{thermal_instability} imply SI requires
% \begin{align*}
%   \frac{\imag\left[\delta\mathcal{C}\nabla\cdot\dd\bm{v}^*\right]}{\real(\sigma)}<0. 
% \end{align*}
% The linearized dust diffusion or `cooling' term $\delta\mathcal{C}$
% for this problem is given by Eq. \ref{lin_drag_si}. Thus the
% requirement is 
% \begin{align*}
% \tepsilon |\bm{k}|^2 
%   \frac{\imag\left(\ii \sigma^* QW^*\right)}{\real(\sigma)}
% -k_x F_r\left( 1-2\tepsilon\right) 
%  | W |^2\frac{\imag\left(\sigma^*\right)}{\real(\sigma)} < 0, 
% \end{align*}
% where we have used Eq. \ref{streaming_mass}. 
% For slow growth and/or large $\left|\bm{k}\right|^2$ the second term 
% is small compared to the first. In this case instablity is due to  


%\subsection{Boundary conditions}
%Eq. \ref{lin_mass}---\ref{lin_energy} can be reduced to a set of
%first-order differential equations for $W, Q$ and $\dd v_z$. We can
%see this schematically as follows. Eq. \ref{lin_xmom} and
%\ref{lin_ymom} may be combined to yield 
%\begin{align*}
%  \dd v_r = \dd v_r (W, Q, \dd v_z). 
%\end{align*} 
%We can then take the continuity equation as an equation for $\dd v_z$, 
%\begin{align*}
%\text{Eq. \ref{lin_mass}} \Rightarrow \dd v_z^\prime(z) = \dd
%  v_z^\prime(W,Q,\dd v_z),
%\end{align*}
%and the vertical momentum equation as an equation for $Q$, 
%\begin{align*}
%\text{Eq. \ref{lin_zmom}} \Rightarrow Q^\prime(z) = 
% Q^\prime(W,Q,\dd v_z). 
%\end{align*}

%Now, for finite dust-gas coupling, $\tstop\neq0$, inspection of $\dd C$
%(Appendix \ref{lin_dust}) shows that it involves $W^\prime$ (assuming
%$\tepsilon < 1$). Then Eq. \ref{lin_energy} may be taken as a
%differential equation for $W$. However, for perfectly coupled dust,
%$\tstop =\dd\mathcal{C}= 0$. In that case Eq. \ref{lin_energy} gives an algebraic
%relation between $W$, $Q$ and $\dd v_z$. That is,
%\begin{align*}
%\text{Eq. \ref{lin_energy}}\Rightarrow \begin{cases}
%  W^\prime(z) = W^\prime(W, Q, \dd v_z) & \text{if } \tstop \neq 0, \\
%  W           (z) = W(Q, \dd v_z) & \text{if }\tstop = 0.
%\end{cases}
%\end{align*} 

%This means that for the perfectly coupled problem, we have a pair of
%ODEs for $Q$ and $\dd v_z$, and 
%When
%$\tstop\neq0$, we require three boundary conditions. {\bf WHAT?? Why odd
%  number of BCs? Shouldn't we have even number of BCs? We have
%two boundaries? I've only seen problems like this involving even
%number of BCs.} In this case we impose $\dd v_z =0$ at $z=\pm
%z_\mathrm{max}$, and classify modes according to their symmetry about
%the midplane: `even' modes with $\dd v_z^\prime(0) = 0$, and `odd' 
%modes with $\dd v_z(0)=0$.       

